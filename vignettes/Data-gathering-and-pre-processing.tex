% Options for packages loaded elsewhere
\PassOptionsToPackage{unicode}{hyperref}
\PassOptionsToPackage{hyphens}{url}
%
\documentclass[
]{article}
\usepackage{amsmath,amssymb}
\usepackage{lmodern}
\usepackage{iftex}
\ifPDFTeX
  \usepackage[T1]{fontenc}
  \usepackage[utf8]{inputenc}
  \usepackage{textcomp} % provide euro and other symbols
\else % if luatex or xetex
  \usepackage{unicode-math}
  \defaultfontfeatures{Scale=MatchLowercase}
  \defaultfontfeatures[\rmfamily]{Ligatures=TeX,Scale=1}
\fi
% Use upquote if available, for straight quotes in verbatim environments
\IfFileExists{upquote.sty}{\usepackage{upquote}}{}
\IfFileExists{microtype.sty}{% use microtype if available
  \usepackage[]{microtype}
  \UseMicrotypeSet[protrusion]{basicmath} % disable protrusion for tt fonts
}{}
\makeatletter
\@ifundefined{KOMAClassName}{% if non-KOMA class
  \IfFileExists{parskip.sty}{%
    \usepackage{parskip}
  }{% else
    \setlength{\parindent}{0pt}
    \setlength{\parskip}{6pt plus 2pt minus 1pt}}
}{% if KOMA class
  \KOMAoptions{parskip=half}}
\makeatother
\usepackage{xcolor}
\usepackage[margin=1in]{geometry}
\usepackage{color}
\usepackage{fancyvrb}
\newcommand{\VerbBar}{|}
\newcommand{\VERB}{\Verb[commandchars=\\\{\}]}
\DefineVerbatimEnvironment{Highlighting}{Verbatim}{commandchars=\\\{\}}
% Add ',fontsize=\small' for more characters per line
\usepackage{framed}
\definecolor{shadecolor}{RGB}{248,248,248}
\newenvironment{Shaded}{\begin{snugshade}}{\end{snugshade}}
\newcommand{\AlertTok}[1]{\textcolor[rgb]{0.94,0.16,0.16}{#1}}
\newcommand{\AnnotationTok}[1]{\textcolor[rgb]{0.56,0.35,0.01}{\textbf{\textit{#1}}}}
\newcommand{\AttributeTok}[1]{\textcolor[rgb]{0.77,0.63,0.00}{#1}}
\newcommand{\BaseNTok}[1]{\textcolor[rgb]{0.00,0.00,0.81}{#1}}
\newcommand{\BuiltInTok}[1]{#1}
\newcommand{\CharTok}[1]{\textcolor[rgb]{0.31,0.60,0.02}{#1}}
\newcommand{\CommentTok}[1]{\textcolor[rgb]{0.56,0.35,0.01}{\textit{#1}}}
\newcommand{\CommentVarTok}[1]{\textcolor[rgb]{0.56,0.35,0.01}{\textbf{\textit{#1}}}}
\newcommand{\ConstantTok}[1]{\textcolor[rgb]{0.00,0.00,0.00}{#1}}
\newcommand{\ControlFlowTok}[1]{\textcolor[rgb]{0.13,0.29,0.53}{\textbf{#1}}}
\newcommand{\DataTypeTok}[1]{\textcolor[rgb]{0.13,0.29,0.53}{#1}}
\newcommand{\DecValTok}[1]{\textcolor[rgb]{0.00,0.00,0.81}{#1}}
\newcommand{\DocumentationTok}[1]{\textcolor[rgb]{0.56,0.35,0.01}{\textbf{\textit{#1}}}}
\newcommand{\ErrorTok}[1]{\textcolor[rgb]{0.64,0.00,0.00}{\textbf{#1}}}
\newcommand{\ExtensionTok}[1]{#1}
\newcommand{\FloatTok}[1]{\textcolor[rgb]{0.00,0.00,0.81}{#1}}
\newcommand{\FunctionTok}[1]{\textcolor[rgb]{0.00,0.00,0.00}{#1}}
\newcommand{\ImportTok}[1]{#1}
\newcommand{\InformationTok}[1]{\textcolor[rgb]{0.56,0.35,0.01}{\textbf{\textit{#1}}}}
\newcommand{\KeywordTok}[1]{\textcolor[rgb]{0.13,0.29,0.53}{\textbf{#1}}}
\newcommand{\NormalTok}[1]{#1}
\newcommand{\OperatorTok}[1]{\textcolor[rgb]{0.81,0.36,0.00}{\textbf{#1}}}
\newcommand{\OtherTok}[1]{\textcolor[rgb]{0.56,0.35,0.01}{#1}}
\newcommand{\PreprocessorTok}[1]{\textcolor[rgb]{0.56,0.35,0.01}{\textit{#1}}}
\newcommand{\RegionMarkerTok}[1]{#1}
\newcommand{\SpecialCharTok}[1]{\textcolor[rgb]{0.00,0.00,0.00}{#1}}
\newcommand{\SpecialStringTok}[1]{\textcolor[rgb]{0.31,0.60,0.02}{#1}}
\newcommand{\StringTok}[1]{\textcolor[rgb]{0.31,0.60,0.02}{#1}}
\newcommand{\VariableTok}[1]{\textcolor[rgb]{0.00,0.00,0.00}{#1}}
\newcommand{\VerbatimStringTok}[1]{\textcolor[rgb]{0.31,0.60,0.02}{#1}}
\newcommand{\WarningTok}[1]{\textcolor[rgb]{0.56,0.35,0.01}{\textbf{\textit{#1}}}}
\usepackage{graphicx}
\makeatletter
\def\maxwidth{\ifdim\Gin@nat@width>\linewidth\linewidth\else\Gin@nat@width\fi}
\def\maxheight{\ifdim\Gin@nat@height>\textheight\textheight\else\Gin@nat@height\fi}
\makeatother
% Scale images if necessary, so that they will not overflow the page
% margins by default, and it is still possible to overwrite the defaults
% using explicit options in \includegraphics[width, height, ...]{}
\setkeys{Gin}{width=\maxwidth,height=\maxheight,keepaspectratio}
% Set default figure placement to htbp
\makeatletter
\def\fps@figure{htbp}
\makeatother
\setlength{\emergencystretch}{3em} % prevent overfull lines
\providecommand{\tightlist}{%
  \setlength{\itemsep}{0pt}\setlength{\parskip}{0pt}}
\setcounter{secnumdepth}{-\maxdimen} % remove section numbering
\newlength{\cslhangindent}
\setlength{\cslhangindent}{1.5em}
\newlength{\csllabelwidth}
\setlength{\csllabelwidth}{3em}
\newlength{\cslentryspacingunit} % times entry-spacing
\setlength{\cslentryspacingunit}{\parskip}
\newenvironment{CSLReferences}[2] % #1 hanging-ident, #2 entry spacing
 {% don't indent paragraphs
  \setlength{\parindent}{0pt}
  % turn on hanging indent if param 1 is 1
  \ifodd #1
  \let\oldpar\par
  \def\par{\hangindent=\cslhangindent\oldpar}
  \fi
  % set entry spacing
  \setlength{\parskip}{#2\cslentryspacingunit}
 }%
 {}
\usepackage{calc}
\newcommand{\CSLBlock}[1]{#1\hfill\break}
\newcommand{\CSLLeftMargin}[1]{\parbox[t]{\csllabelwidth}{#1}}
\newcommand{\CSLRightInline}[1]{\parbox[t]{\linewidth - \csllabelwidth}{#1}\break}
\newcommand{\CSLIndent}[1]{\hspace{\cslhangindent}#1}

\usepackage{tcolorbox}

\definecolor{mybluebackground}{HTML}{e7f3fe}
\definecolor{mybluecolor}{HTML}{31708f}
\definecolor{myblueborder}{HTML}{bce8f1}
\definecolor{mybrownbackground}{HTML}{f9f0e4}
\definecolor{mybrowncolor}{HTML}{8a6d3b}
\definecolor{mybrownborder}{HTML}{faebcc}

\newtcolorbox{infobox}{
  colback=mybluebackground,
  colframe=myblueborder,
  coltext=mybluecolor,
  boxsep=3pt,
  boxrule=1pt,
  arc=4pt}

\newtcolorbox{warningbox}{
  colback=mybrownbackground,
  colframe=mybrownborder,
  coltext=mybrowncolor,
  boxsep=3pt,
  boxrule=1pt,
  arc=4pt}

\usepackage{amsmath}
\usepackage{booktabs}
\usepackage{float}
\usepackage{subcaption}
\usepackage{caption}
\floatplacement{figure}{H}
\ifLuaTeX
  \usepackage{selnolig}  % disable illegal ligatures
\fi
\IfFileExists{bookmark.sty}{\usepackage{bookmark}}{\usepackage{hyperref}}
\IfFileExists{xurl.sty}{\usepackage{xurl}}{} % add URL line breaks if available
\urlstyle{same} % disable monospaced font for URLs
\hypersetup{
  pdftitle={Data gathering and (pre-)processing},
  pdfauthor={Ian Ondo},
  hidelinks,
  pdfcreator={LaTeX via pandoc}}

\title{Data gathering and (pre-)processing}
\author{Ian Ondo}
\date{2023-05-29}

\begin{document}
\maketitle

{
\setcounter{tocdepth}{2}
\tableofcontents
}
\begin{warningbox}

\textbf{Package notes:}\\
We need to install the following set of R packages to successfully run
the codes in this vignette:

\begin{itemize}
\tightlist
\item
  \textbf{data.table} (Dowle and Srinivasan 2021)
\item
  \textbf{rWCVP} (Brown et al. 2023)
\end{itemize}

\end{warningbox}

\hypertarget{introduction}{%
\section{Introduction}\label{introduction}}

In this vignette, we will show how to compile and format occurrence data
from various data sources available online, or as zipped (compressed)
files using \textbf{UsefulPlants}. We will focus on two online
databases:

\begin{itemize}
\item
  \emph{Global Biodiversity Information Facility} (\textbf{GBIF},
  \href{https://www.gbif.org}{\emph{\underline{https://www.gbif.org}}})
\item
  \emph{Botanical information and Ecology Network} (\textbf{BIEN version
  4.1},
  \href{https://bien.nceas.ucsb.edu/bien}{\emph{\underline{https://bien.nceas.ucsb.edu/bien}}})
\end{itemize}

Both have R packages facilitating the download of occurrence records
directly from R. For example, the function \emph{get\_gbifid\_} from the
R package \textbf{taxize} (Scott Chamberlain and Eduard Szocs 2013) and
function \emph{occ\_download} from the R package \textbf{rgbif}
(Chamberlain and Boettiger 2017) can be used for matching species names
and downloading occurrence records from GBIF respectively, while the
function \emph{BIEN\_occurrence\_species} from the R package
\textbf{BIEN} (Maitner 2020)helps to extract occurrence records from the
BIEN database.\\
To our knowledge, other databases such as \textbf{RAINBIO}
(\href{http://rainbio.cesab.org/}{\underline{http://rainbio.cesab.org/}}),
\textbf{speciesLink}
(\href{http://www.splink.org.br/}{\underline{http://www.splink.org.br/}}),
\textbf{BioTIME}
(\href{https://biotime.st-andrews.ac.uk/}{\underline{https://biotime.st-andrews.ac.uk/}})
or \textbf{Genesys}
(\href{https://www.genesys-pgr.org}{\underline{https://www.genesys-pgr.org}})
do not provide R packages to access their data, but snapshots of the
entire database can be accessed, sometimes on request, and often shared
as a (compressed) file. In this case the \textbf{data.table} (Dowle and
Srinivasan 2021) is a very useful R package in terms of efficiency and
speed to process these type of datasets.

\hypertarget{i.-download-a-list-of-useful-plants-species-names}{%
\section{I. Download a list of useful plants species
names}\label{i.-download-a-list-of-useful-plants-species-names}}

Download the useful plants database used in Pironon and Ondo et al.~2023
\href{https://www.here.com}{\textcolor{blue}{\underline{here}}}, and
unzip it into a folder, e.g.~\emph{``usefulplant\_data''}. Then, read in
the data and filter as needed:

\begin{Shaded}
\begin{Highlighting}[]

\CommentTok{\# load the data into memory}
\NormalTok{useful\_plant\_db }\OtherTok{\textless{}{-}}\NormalTok{ data.table}\SpecialCharTok{::}\FunctionTok{fread}\NormalTok{(}\AttributeTok{input=}\StringTok{"usefulplant\_data/useful\_plants\_data\_list.csv"}\NormalTok{,}
                                     \AttributeTok{h=}\ConstantTok{TRUE}\NormalTok{,}
                                     \AttributeTok{showProgress=}\ConstantTok{FALSE}\NormalTok{)}

\NormalTok{useful\_plant\_db}

\CommentTok{\# get the list of all useful plants names in the database}
\NormalTok{useful\_plant\_names }\OtherTok{\textless{}{-}}\NormalTok{ useful\_plant\_db }\SpecialCharTok{\%\textgreater{}\%}
\NormalTok{  dplyr}\SpecialCharTok{::}\FunctionTok{pull}\NormalTok{(binomial\_acc\_name) }\CommentTok{\# extract binomial accepted names}

\CommentTok{\# optionally filter e.g. edible plants}
\NormalTok{edible\_plant\_names }\OtherTok{\textless{}{-}}\NormalTok{ useful\_plant\_db }\SpecialCharTok{\%\textgreater{}\%}
\NormalTok{  dplyr}\SpecialCharTok{::}\FunctionTok{filter}\NormalTok{(HumanFood}\SpecialCharTok{==}\DecValTok{1}\NormalTok{) }\SpecialCharTok{\%\textgreater{}\%}         \CommentTok{\# select edible plants}
\NormalTok{  dplyr}\SpecialCharTok{::}\FunctionTok{pull}\NormalTok{(binomial\_acc\_name)          }
\end{Highlighting}
\end{Shaded}

\hypertarget{ii.-download-occurrence-data-from-gbif}{%
\section{II. Download occurrence data from
GBIF}\label{ii.-download-occurrence-data-from-gbif}}

\textbf{UsefulPlants} has a function called \texttt{queryGBIF} that uses
both \emph{get\_gbifid\_} and \emph{occ\_download} to obtain occurrence
records from GBIF from a list of species names. The function
\texttt{get\_gbiftaxonkey} is used internally to get taxon \emph{keys},
and the essential of the code is described in this blog
\href{https://data-blog.gbif.org/post/downloading-long-species-lists-on-gbif/}{\emph{\textcolor{blue}{\underline{downloading a long species list on GBIF}}}},
feel free to have a look at it.\\
Basically, we first need to create a GBIF account because we are going
to use GBIF credentials as explained in the blog to login to GBIF and
download occurrence records for a fairly large number of species.

\hypertarget{ii.1-create-a-gbif-account}{%
\subsection{\texorpdfstring{II.1 \emph{create a GBIF
account}}{II.1 create a GBIF account}}\label{ii.1-create-a-gbif-account}}

Go to the GBIF website homepage
\href{https://www.gbif.org}{\underline{https://www.gbif.org}}
\rightarrow click on the \texttt{login} button on the top-right corner
of the page \rightarrow click on \texttt{register} \rightarrow fill up
the fields to create your account. Then, save your
\emph{\underline{GBIF username}}, \emph{\underline{email}} associated
with username and \emph{\underline{password}} for logging in to GBIF
somewhere safe, for example in your .Renviron file:

\begin{Shaded}
\begin{Highlighting}[]
\FunctionTok{install.packages}\NormalTok{(}\StringTok{"usethis"}\NormalTok{) }\CommentTok{\# if not yet installed}
\FunctionTok{library}\NormalTok{(usethis)}

\CommentTok{\# open your .Renviron file}
\FunctionTok{edit\_r\_environ}\NormalTok{()}

\CommentTok{\# write down your GBIF credentials as environmental variables}
\NormalTok{GBIF\_USER }\OtherTok{=} \StringTok{"your\_username"}
\NormalTok{GBIF\_EMAIL }\OtherTok{=} \StringTok{"your\_email\_address"}
\NormalTok{GBIF\_PWD }\OtherTok{=} \StringTok{"your\_password"}

\CommentTok{\# save your changes, close the file and restart your R session}
\end{Highlighting}
\end{Shaded}

Check that your GBIF credentials are now accessible by typing in the
console e.g.~\texttt{Sys.getenv("GBIF\_USER")}.

\hypertarget{ii.2-use-querygbif-to-download-occurrence-records}{%
\subsection{\texorpdfstring{II.2 \emph{use \texttt{queryGBIF} to
download occurrence
records}}{II.2 use queryGBIF to download occurrence records}}\label{ii.2-use-querygbif-to-download-occurrence-records}}

\begin{Shaded}
\begin{Highlighting}[]

\CommentTok{\# load the useful plants package}
\FunctionTok{library}\NormalTok{(UsefulPlants)}

\CommentTok{\# make sure species names are correctly formatted }
\NormalTok{edible\_plant\_names }\OtherTok{\textless{}{-}} \FunctionTok{gsub}\NormalTok{(}\StringTok{"\_"}\NormalTok{,}\StringTok{" "}\NormalTok{,edible\_plant\_names)}

\CommentTok{\# specify a folder where to store the zip file of your downloads}
\CommentTok{\# it is recommended to use a specific folder for better traceability}
\NormalTok{dwnld\_dir }\OtherTok{=} \FunctionTok{tempdir}\NormalTok{()}
  
\CommentTok{\# run the query using the vector of plant species names}
\NormalTok{gbifDATA }\OtherTok{=} \FunctionTok{queryGBIF}\NormalTok{(}\AttributeTok{species\_name=}\NormalTok{edible\_plant\_names,}
                \AttributeTok{gbif\_download\_dir=}\NormalTok{dwnld\_dir,}
                \CommentTok{\# time in seconds before dropping out the connection}
                \AttributeTok{time\_out =} \DecValTok{300}\NormalTok{)}
\end{Highlighting}
\end{Shaded}

\begin{infobox}

By default, \textbf{queryGBIF} will return after \texttt{time\_out}
seconds. If GBIF has not finished processing your query or if something
went wrong, \texttt{queryGBIF} will returns \texttt{NULL} and an error
message, in this case:

\begin{itemize}
\item
  \textbf{\underline{Login}} to your account on GBIF.
\item
  Go to your downloads and \textbf{\underline{check}} out the status of
  your query.
\item
  Once available, you can manually \textbf{\underline{download}} your
  dataset as zip file in the directory of your choice.
\end{itemize}

\end{infobox}

If everything worked, the output of \texttt{queryGBIF} is a
\texttt{data.table} object with your occurrence records with the
following columns:

\begin{itemize}
\item
  \emph{``species''}
\item
  \emph{``fullname''},
\item
  \emph{``decimalLongitude''}
\item
  \emph{``decimalLatitude''}
\item
  \emph{``countryCode''}
\item
  \emph{``coordinateUncertaintyInMeters''}
\item
  \emph{``year''}
\item
  \emph{``individualCount''}
\item
  \emph{``gbifID''}
\item
  \emph{``basisOfRecord''}
\item
  \emph{``institutionCode''}
\item
  \emph{``establishmentMeans''}
\item
  \emph{``is\_cultivated\_observation''}
\item
  \emph{``sourceID''} (always GBIF)
\end{itemize}

Find documentation on most of the fields
\href{https://www.gbif.org/data-quality-requirements-occurrences}{\emph{https://www.gbif.org/data-quality-requirements-occurrences}}.

If your automatic download failed and you ended up downloading a zip
file manually, you can still format your dataset as follows:

\begin{Shaded}
\begin{Highlighting}[]

\CommentTok{\# get the path to your zip file}
\NormalTok{path\_to\_my\_zip.file }\OtherTok{\textless{}{-}} \StringTok{"path/to/yourzip.file"}

\CommentTok{\# read in the data from your zip file}
\NormalTok{gbifDATA }\OtherTok{\textless{}{-}}\NormalTok{ data.table}\SpecialCharTok{::}\FunctionTok{fread}\NormalTok{(}\AttributeTok{cmd=}\FunctionTok{paste}\NormalTok{(}\StringTok{"unzip {-}p"}\NormalTok{,path\_to\_my\_zip.file),}
                                       \AttributeTok{header=}\ConstantTok{TRUE}\NormalTok{,}
                                       \AttributeTok{showProgress=}\ConstantTok{FALSE}\NormalTok{,}
                                       \AttributeTok{na.strings=}\FunctionTok{c}\NormalTok{(}\StringTok{""}\NormalTok{,}\ConstantTok{NA}\NormalTok{),}
                                       \AttributeTok{fill=}\ConstantTok{FALSE}\NormalTok{,}
                                       \AttributeTok{quote =} \StringTok{""}\NormalTok{,}
                                       \AttributeTok{select=}\FunctionTok{c}\NormalTok{(}\StringTok{"species"}\NormalTok{,}
                                               \StringTok{"taxonRank"}\NormalTok{,}
                                               \StringTok{"infraspecificEpithet"}\NormalTok{,}
                                               \StringTok{"decimalLongitude"}\NormalTok{,}
                                               \StringTok{"decimalLatitude"}\NormalTok{,}
                                               \StringTok{"countryCode"}\NormalTok{,}
                                               \StringTok{"coordinateUncertaintyInMeters"}\NormalTok{,}
                                               \StringTok{"year"}\NormalTok{,}
                                               \StringTok{"gbifID"}\NormalTok{,}
                                               \StringTok{"basisOfRecord"}\NormalTok{,}
                                               \StringTok{"institutionCode"}\NormalTok{,}
                                               \StringTok{"establishmentMeans"}\NormalTok{,}
                                               \StringTok{"individualCount"}\NormalTok{))}

\NormalTok{gbifDATA}\OtherTok{\textless{}{-}}\FunctionTok{na.omit}\NormalTok{(gbifDATA, }\AttributeTok{cols=} \FunctionTok{c}\NormalTok{(}\StringTok{"decimalLatitude"}\NormalTok{, }\StringTok{"decimalLongitude"}\NormalTok{))}

\CommentTok{\# format the data:}
\NormalTok{gbifDATA[, }\StringTok{\textasciigrave{}}\AttributeTok{:=}\StringTok{\textasciigrave{}}\NormalTok{(}\AttributeTok{taxonRank =} \FunctionTok{ifelse}\NormalTok{(taxonRank }\SpecialCharTok{\%in\%} \FunctionTok{c}\NormalTok{(}\StringTok{"SPECIES"}\NormalTok{,}\StringTok{"GENUS"}\NormalTok{),}
                                   \ConstantTok{NA}\NormalTok{,}
                                   \FunctionTok{ifelse}\NormalTok{(taxonRank}\SpecialCharTok{==}\StringTok{"FORM"}\NormalTok{, }\StringTok{"f."}\NormalTok{,}
                                          \FunctionTok{ifelse}\NormalTok{(taxonRank}\SpecialCharTok{==}\StringTok{"SUBSPECIES"}\NormalTok{, }\StringTok{"subsp."}\NormalTok{,}
                                                 \FunctionTok{ifelse}\NormalTok{(taxonRank}\SpecialCharTok{==}\StringTok{"VARIETY"}\NormalTok{,}\StringTok{"var."}\NormalTok{,}
\NormalTok{                                                        taxonRank)))),}
                    \AttributeTok{countryCode =}\NormalTok{ countrycode}\SpecialCharTok{::}\FunctionTok{countrycode}\NormalTok{(countryCode,}
                                                           \AttributeTok{origin =}  \StringTok{\textquotesingle{}iso2c\textquotesingle{}}\NormalTok{,}
                                                           \AttributeTok{destination =} \StringTok{\textquotesingle{}iso3c\textquotesingle{}}\NormalTok{,}
                                                           \AttributeTok{nomatch =} \ConstantTok{NA}\NormalTok{),}
                    \AttributeTok{establishmentMeans =} \FunctionTok{ifelse}\NormalTok{(establishmentMeans}\SpecialCharTok{==}\StringTok{"INTRODUCED"}\NormalTok{,}
                                                \StringTok{"Introduced"}\NormalTok{,}
                                                \FunctionTok{ifelse}\NormalTok{(establishmentMeans}\SpecialCharTok{==} \StringTok{"NATIVE"}\NormalTok{,}
                                                       \StringTok{"Native"}\NormalTok{,}
\NormalTok{                                                       establishmentMeans)))]}

\CommentTok{\# create \textquotesingle{}fullname\textquotesingle{}, \textquotesingle{}is\_cultivated\_observation\textquotesingle{} and \textquotesingle{}sourceID\textquotesingle{} columns}
\NormalTok{gbifDATA[, }\StringTok{\textasciigrave{}}\AttributeTok{:=}\StringTok{\textasciigrave{}}\NormalTok{(}\AttributeTok{fullname =} \FunctionTok{ifelse}\NormalTok{(}\FunctionTok{is.na}\NormalTok{(infraspecificEpithet),}
                                  \FunctionTok{paste}\NormalTok{(species),}
                                  \FunctionTok{paste}\NormalTok{(species, taxonRank, infraspecificEpithet)),}
                \AttributeTok{is\_cultivated\_observation =} \ConstantTok{NA}\NormalTok{,}
                \AttributeTok{sourceID =} \StringTok{\textquotesingle{}GBIF\textquotesingle{}}\NormalTok{)]}

\CommentTok{\# delete taxonRank and infraspecificEpithet columns}
\NormalTok{gbifDATA[, }\StringTok{\textasciigrave{}}\AttributeTok{:=}\StringTok{\textasciigrave{}}\NormalTok{(}\AttributeTok{taxonRank =} \ConstantTok{NULL}\NormalTok{, }\AttributeTok{infraspecificEpithet =} \ConstantTok{NULL}\NormalTok{)]}

\CommentTok{\# remove fossil records}
\NormalTok{gbifDATA }\OtherTok{\textless{}{-}}\NormalTok{ gbifDATA[basisOfRecord}\SpecialCharTok{!=}\StringTok{"FOSSIL\_SPECIMEN"}\NormalTok{]}

\CommentTok{\# set column order}
\NormalTok{colNames }\OtherTok{=} \FunctionTok{c}\NormalTok{(}\StringTok{"species"}\NormalTok{,}
             \StringTok{"fullname"}\NormalTok{,}
             \StringTok{"decimalLongitude"}\NormalTok{,}
             \StringTok{"decimalLatitude"}\NormalTok{,}
             \StringTok{"countryCode"}\NormalTok{,}
             \StringTok{"coordinateUncertaintyInMeters"}\NormalTok{,}
             \StringTok{"year"}\NormalTok{,}
             \StringTok{"individualCount"}\NormalTok{,}
             \StringTok{"gbifID"}\NormalTok{,}
             \StringTok{"basisOfRecord"}\NormalTok{,}
             \StringTok{"institutionCode"}\NormalTok{,}
             \StringTok{"establishmentMeans"}\NormalTok{,}
             \StringTok{"is\_cultivated\_observation"}\NormalTok{,}
             \StringTok{"sourceID"}\NormalTok{)}

\NormalTok{data.table}\SpecialCharTok{::}\FunctionTok{setcolorder}\NormalTok{(gbifDATA, colNames)}

\CommentTok{\# set the key to the species column to enable fast binary search}
\NormalTok{data.table}\SpecialCharTok{::}\FunctionTok{setkey}\NormalTok{(gbifDATA, }\StringTok{\textquotesingle{}species\textquotesingle{}}\NormalTok{)}
\end{Highlighting}
\end{Shaded}

\hypertarget{iii.-download-occurrence-data-from-bien}{%
\section{III. Download occurrence data from
BIEN}\label{iii.-download-occurrence-data-from-bien}}

Similarly, \textbf{UsefulPlants} has a function called
\texttt{queryBIEN} to download occurrence records from the BIEN
database. Internally, \texttt{queryBIEN} uses
\texttt{BIEN\_occurrence\_species} with the following default arguments:

\begin{Shaded}
\begin{Highlighting}[]
\NormalTok{cultivated }\OtherTok{=} \ConstantTok{TRUE}\NormalTok{,}
\NormalTok{only.new.world }\OtherTok{=} \ConstantTok{FALSE}\NormalTok{,}
\NormalTok{all.taxonomy }\OtherTok{=} \ConstantTok{TRUE}\NormalTok{,}
\NormalTok{native.status }\OtherTok{=} \ConstantTok{TRUE}\NormalTok{,}
\NormalTok{observation.type }\OtherTok{=} \ConstantTok{TRUE}\NormalTok{,}
\NormalTok{political.boundaries }\OtherTok{=} \ConstantTok{TRUE}\NormalTok{,}
\NormalTok{natives.only }\OtherTok{=} \ConstantTok{FALSE}\NormalTok{,}
\NormalTok{collection.info }\OtherTok{=} \ConstantTok{TRUE}
\end{Highlighting}
\end{Shaded}

Check out the meaning of these arguments from the package help here
\href{https://search.r-project.org/CRAN/refmans/BIEN/html/BIEN_occurrence_species.html}{\emph{\textcolor{blue}{\underline{manual}}}}

\begin{Shaded}
\begin{Highlighting}[]

\CommentTok{\# run the query using the vector of plant species names previously defined in section II.2 }
\NormalTok{bienDATA }\OtherTok{=} \FunctionTok{queryBIEN}\NormalTok{(}\AttributeTok{species\_name=}\NormalTok{edible\_plant\_names)}
\end{Highlighting}
\end{Shaded}

If everything works fine, \texttt{bienDATA} should be a
\texttt{data.table} object formatted as \texttt{gbifDATA}, i.e.~with the
same column names.

\hypertarget{iv.-formatting-occurrence-data-from-rainbio-specieslink-biotime-and-genesys}{%
\section{\texorpdfstring{IV. Formatting occurrence data from
\emph{RAINBIO}, \emph{speciesLink}, \emph{BioTIME} and
\emph{Genesys}}{IV. Formatting occurrence data from RAINBIO, speciesLink, BioTIME and Genesys}}\label{iv.-formatting-occurrence-data-from-rainbio-specieslink-biotime-and-genesys}}

For consistency between the different databases, we need to format each
occurrence dataset the same way as we did for \emph{GBIF} and
\emph{BIEN}.\\
Let's assume that each dataset comes as a zipped folder, if not the
case, replace \texttt{paste("unzip\ -\ p","path/to/file.zip")} by the
path to your unzipped file in the relevant part of the code.

\begin{Shaded}
\begin{Highlighting}[]
\CommentTok{\# define column names}
\NormalTok{colNames }\OtherTok{=} \FunctionTok{c}\NormalTok{(}\StringTok{"species"}\NormalTok{,}
             \StringTok{"fullname"}\NormalTok{,}
             \StringTok{"decimalLongitude"}\NormalTok{,}
             \StringTok{"decimalLatitude"}\NormalTok{,}
             \StringTok{"countryCode"}\NormalTok{,}
             \StringTok{"coordinateUncertaintyInMeters"}\NormalTok{,}
             \StringTok{"year"}\NormalTok{,}
             \StringTok{"individualCount"}\NormalTok{,}
             \StringTok{"gbifID"}\NormalTok{,}
             \StringTok{"basisOfRecord"}\NormalTok{,}
             \StringTok{"institutionCode"}\NormalTok{,}
             \StringTok{"establishmentMeans"}\NormalTok{,}
             \StringTok{"is\_cultivated\_observation"}\NormalTok{,}
             \StringTok{"sourceID"}\NormalTok{)}
\end{Highlighting}
\end{Shaded}

\hypertarget{rainbio}{%
\subsection{\texorpdfstring{\emph{RAINBIO}}{RAINBIO}}\label{rainbio}}

\begin{Shaded}
\begin{Highlighting}[]

\CommentTok{\# read in the data}
\NormalTok{rainbioDATA }\OtherTok{\textless{}{-}}\NormalTok{ data.table}\SpecialCharTok{::}\FunctionTok{fread}\NormalTok{(}\FunctionTok{paste}\NormalTok{(}\StringTok{"unzip {-}p"}\NormalTok{,}\StringTok{"path/to/file.zip"}\NormalTok{),}
                                     \AttributeTok{header=}\ConstantTok{TRUE}\NormalTok{,}
                                     \AttributeTok{showProgress=}\ConstantTok{FALSE}\NormalTok{,}
                                     \AttributeTok{select=}\FunctionTok{c}\NormalTok{(}\StringTok{"tax\_sp\_level"}\NormalTok{,}
                                      \StringTok{"species"}\NormalTok{,}
                                      \StringTok{"decimalLatitude"}\NormalTok{,}
                                      \StringTok{"decimalLongitude"}\NormalTok{,}
                                      \StringTok{"iso3lonlat"}\NormalTok{,}
                                      \StringTok{"basisOfRecord"}\NormalTok{,}
                                      \StringTok{"institutionCode"}\NormalTok{,}
                                      \StringTok{"catalogNumber"}\NormalTok{,}
                                      \StringTok{"coly"}\NormalTok{))}

\CommentTok{\#{-}{-}{-}{-}{-}{-}{-}{-}{-}{-}{-}{-}}
\CommentTok{\#= formatting}
\CommentTok{\#{-}{-}{-}{-}{-}{-}{-}{-}{-}{-}{-}{-}}
\CommentTok{\# change the column names}
\NormalTok{data.table}\SpecialCharTok{::}\FunctionTok{setnames}\NormalTok{(rainbioDATA, }
                     \FunctionTok{c}\NormalTok{(}\StringTok{"species"}\NormalTok{,}
                       \StringTok{"fullname"}\NormalTok{,}
                       \StringTok{"decimalLatitude"}\NormalTok{,}
                       \StringTok{"decimalLongitude"}\NormalTok{,}
                       \StringTok{"countryCode"}\NormalTok{,}
                       \StringTok{"basisOfRecord"}\NormalTok{,}
                       \StringTok{"institutionCode"}\NormalTok{,}
                       \StringTok{"gbifID"}\NormalTok{,}
                       \StringTok{"year"}\NormalTok{))}

\CommentTok{\# remove NA coordinates}
\NormalTok{rainbioDATA }\OtherTok{\textless{}{-}} \FunctionTok{na.omit}\NormalTok{(rainbioDATA, }\AttributeTok{cols=} \FunctionTok{c}\NormalTok{(}\StringTok{"decimalLatitude"}\NormalTok{, }\StringTok{"decimalLongitude"}\NormalTok{))}

\CommentTok{\# create additional columns: \textquotesingle{}coordinateUncertaintyInMeters\textquotesingle{} , \textquotesingle{}is\_cultivated\_observation\textquotesingle{},}
\CommentTok{\# \textquotesingle{}establishmentMeans\textquotesingle{}, \textquotesingle{}individualCount\textquotesingle{} and \textquotesingle{}sourceID\textquotesingle{}}

\NormalTok{rainbioDATA[ ,}\StringTok{\textasciigrave{}}\AttributeTok{:=}\StringTok{\textasciigrave{}}\NormalTok{(}\AttributeTok{coordinateUncertaintyInMeters =} \ConstantTok{NA}\NormalTok{,}
                  \AttributeTok{is\_cultivated\_observation =} \StringTok{"No"}\NormalTok{,}
                  \AttributeTok{establishmentMeans =} \ConstantTok{NA}\NormalTok{,}
                  \AttributeTok{individualCount =} \ConstantTok{NA}\NormalTok{,}
                  \AttributeTok{sourceID =} \StringTok{\textquotesingle{}RAINBIO\textquotesingle{}}\NormalTok{)]}

\CommentTok{\# set column order}
\NormalTok{data.table}\SpecialCharTok{::}\FunctionTok{setcolorder}\NormalTok{(rainbioDATA, colNames)}

\CommentTok{\# set the key to the species column to enable fast binary search}
\NormalTok{data.table}\SpecialCharTok{::}\FunctionTok{setkey}\NormalTok{(rainbioDATA, }\StringTok{\textquotesingle{}species\textquotesingle{}}\NormalTok{)}
\end{Highlighting}
\end{Shaded}

\hypertarget{specieslink}{%
\subsection{\texorpdfstring{\emph{speciesLink}}{speciesLink}}\label{specieslink}}

\begin{Shaded}
\begin{Highlighting}[]

\CommentTok{\# read in the data}
\NormalTok{spLinkDATA }\OtherTok{\textless{}{-}}\NormalTok{ data.table}\SpecialCharTok{::}\FunctionTok{fread}\NormalTok{(}\FunctionTok{paste}\NormalTok{(}\StringTok{"unzip {-}p"}\NormalTok{,}\StringTok{"path/to/file.zip"}\NormalTok{),}
                                    \AttributeTok{header=}\ConstantTok{TRUE}\NormalTok{,}
                                    \AttributeTok{showProgress=}\ConstantTok{FALSE}\NormalTok{,}
                                    \AttributeTok{encoding =} \StringTok{\textquotesingle{}UTF{-}8\textquotesingle{}}\NormalTok{,}
                                    \AttributeTok{na.strings=}\FunctionTok{c}\NormalTok{(}\StringTok{""}\NormalTok{,}\ConstantTok{NA}\NormalTok{),}
                                    \AttributeTok{select=} \FunctionTok{c}\NormalTok{(}\StringTok{"query"}\NormalTok{,}
                                             \StringTok{"scientificname"}\NormalTok{,}
                                             \StringTok{"longitude"}\NormalTok{,}
                                             \StringTok{"latitude"}\NormalTok{,}
                                             \StringTok{"country"}\NormalTok{,}
                                             \StringTok{"coordinateprecision"}\NormalTok{,}
                                             \StringTok{"yearcollected"}\NormalTok{,}
                                             \StringTok{"individualcount"}\NormalTok{,}
                                             \StringTok{"catalognumber"}\NormalTok{,}
                                             \StringTok{"basisofrecord"}\NormalTok{,}
                                             \StringTok{"institutioncode"}\NormalTok{))}

\CommentTok{\#{-}{-}{-}{-}{-}{-}{-}{-}{-}{-}{-}{-}}
\CommentTok{\#= formatting}
\CommentTok{\#{-}{-}{-}{-}{-}{-}{-}{-}{-}{-}{-}{-}}
\CommentTok{\# change the column names}
\NormalTok{data.table}\SpecialCharTok{::}\FunctionTok{setnames}\NormalTok{(spLinkDATA, }
                     \FunctionTok{c}\NormalTok{(}\StringTok{"species"}\NormalTok{,}
                       \StringTok{"fullname"}\NormalTok{,}
                       \StringTok{"decimalLatitude"}\NormalTok{,}
                       \StringTok{"decimalLongitude"}\NormalTok{,}
                       \StringTok{"countryCode"}\NormalTok{,}
                       \StringTok{"coordinateUncertaintyInMeters"}\NormalTok{,}
                       \StringTok{"year"}\NormalTok{,}
                       \StringTok{"individualCount"}\NormalTok{,}
                       \StringTok{"gbifID"}\NormalTok{,}
                       \StringTok{"basisOfRecord"}\NormalTok{,}
                       \StringTok{"institutionCode"}\NormalTok{))}

\CommentTok{\# remove NA coordinates}
\NormalTok{spLinkDATA }\OtherTok{\textless{}{-}} \FunctionTok{na.omit}\NormalTok{(spLinkDATA, }\AttributeTok{cols=} \FunctionTok{c}\NormalTok{(}\StringTok{"decimalLatitude"}\NormalTok{, }\StringTok{"decimalLongitude"}\NormalTok{))}

\NormalTok{spLinkDATA[ ,}\StringTok{\textasciigrave{}}\AttributeTok{:=}\StringTok{\textasciigrave{}}\NormalTok{(}\AttributeTok{species =} \FunctionTok{paste0}\NormalTok{(}\FunctionTok{substr}\NormalTok{(species,}\DecValTok{1}\NormalTok{,}\DecValTok{1}\NormalTok{),}
                                   \FunctionTok{tolower}\NormalTok{(}\FunctionTok{substr}\NormalTok{(species,}\DecValTok{2}\NormalTok{,}\FunctionTok{nchar}\NormalTok{(species)))),}
                 \AttributeTok{countryCode =} \FunctionTok{ifelse}\NormalTok{(countryCode}\SpecialCharTok{==}\StringTok{\textquotesingle{}Brasil\textquotesingle{}}\NormalTok{,}\StringTok{\textquotesingle{}Brazil\textquotesingle{}}\NormalTok{,countryCode),}
                 \AttributeTok{basisOfRecord =} \FunctionTok{ifelse}\NormalTok{(basisOfRecord}\SpecialCharTok{==}\StringTok{"S"}\NormalTok{,}\StringTok{"SPECIMEN"}\NormalTok{,}
                                        \FunctionTok{ifelse}\NormalTok{(basisOfRecord}\SpecialCharTok{==} \StringTok{"O"}\NormalTok{,}
                                               \StringTok{"OBSERVATION"}\NormalTok{,}
\NormalTok{                                               basisOfRecord)),}
                 \AttributeTok{is\_cultivated\_observation =} \ConstantTok{NA}\NormalTok{,}
                 \AttributeTok{establishmentMeans =} \ConstantTok{NA}\NormalTok{,}
                 \AttributeTok{sourceID =} \StringTok{\textquotesingle{}spLink\textquotesingle{}}\NormalTok{)]}

\NormalTok{spLinkDATA[ ,countryCode}\SpecialCharTok{:}\ErrorTok{=}\NormalTok{ countrycode}\SpecialCharTok{::}\FunctionTok{countrycode}\NormalTok{(countryCode,}
                                                    \AttributeTok{origin =}  \StringTok{\textquotesingle{}country.name\textquotesingle{}}\NormalTok{,}
                                                    \AttributeTok{destination =} \StringTok{\textquotesingle{}iso3c\textquotesingle{}}\NormalTok{,}
                                                    \AttributeTok{nomatch =} \ConstantTok{NA}\NormalTok{)]}
\CommentTok{\# set column order}
\NormalTok{data.table}\SpecialCharTok{::}\FunctionTok{setcolorder}\NormalTok{(spLinkDATA, colNames)}

\CommentTok{\# set the key to the species column to enable fast binary search}
\NormalTok{data.table}\SpecialCharTok{::}\FunctionTok{setkey}\NormalTok{(spLinkDATA, }\StringTok{\textquotesingle{}species\textquotesingle{}}\NormalTok{)}
\end{Highlighting}
\end{Shaded}

\hypertarget{biotime}{%
\subsection{\texorpdfstring{\emph{bioTIME}}{bioTIME}}\label{biotime}}

\begin{Shaded}
\begin{Highlighting}[]

\CommentTok{\# read in the data}
\NormalTok{biotimeDATA  }\OtherTok{\textless{}{-}}\NormalTok{ data.table}\SpecialCharTok{::}\FunctionTok{fread}\NormalTok{(}\FunctionTok{paste}\NormalTok{(}\StringTok{"unzip {-}p"}\NormalTok{,}\StringTok{"path/to/file.zip"}\NormalTok{),}
                                  \AttributeTok{header=}\ConstantTok{TRUE}\NormalTok{,}
                                  \AttributeTok{showProgress=}\ConstantTok{FALSE}\NormalTok{,}
                                  \AttributeTok{select=}\FunctionTok{c}\NormalTok{(}\StringTok{"GENUS\_SPECIES"}\NormalTok{,}
                                           \StringTok{"LATITUDE"}\NormalTok{,}
                                           \StringTok{"LONGITUDE"}\NormalTok{,}
                                           \StringTok{"YEAR"}\NormalTok{,}
                                           \StringTok{"sum.allrawdata.ABUNDANCE"}\NormalTok{,}
                                           \StringTok{"SAMPLE\_DESC"}\NormalTok{))}

\CommentTok{\#{-}{-}{-}{-}{-}{-}{-}{-}{-}{-}{-}{-}{-}{-}{-}}
\CommentTok{\#= formatting}
\CommentTok{\#{-}{-}{-}{-}{-}{-}{-}{-}{-}{-}{-}{-}{-}{-}{-}}

\CommentTok{\# change the column names}
\NormalTok{data.table}\SpecialCharTok{::}\FunctionTok{setnames}\NormalTok{(biotimeDATA, }\FunctionTok{c}\NormalTok{(}\StringTok{"species"}\NormalTok{,}
                                    \StringTok{"decimalLatitude"}\NormalTok{,}
                                    \StringTok{"decimalLongitude"}\NormalTok{,}
                                    \StringTok{"year"}\NormalTok{,}
                                    \StringTok{"individualCount"}\NormalTok{,}
                                    \StringTok{"gbifID"}\NormalTok{))}

\CommentTok{\# remove NA coordinates}
\NormalTok{biotimeDATA  }\OtherTok{\textless{}{-}} \FunctionTok{na.omit}\NormalTok{(biotimeDATA, }\AttributeTok{cols=} \FunctionTok{c}\NormalTok{(}\StringTok{"decimalLatitude"}\NormalTok{, }\StringTok{"decimalLongitude"}\NormalTok{))}

\CommentTok{\# create additional columns: \textquotesingle{}fullname\textquotesingle{}, \textquotesingle{}coordinateUncertaintyInMeters\textquotesingle{},\textquotesingle{}institutionCode\textquotesingle{},}
\CommentTok{\#\textquotesingle{}countryCode\textquotesingle{}, \textquotesingle{}basisOfRecord\textquotesingle{}, \textquotesingle{}establishmentMeans\textquotesingle{}, \textquotesingle{}is\_cultivated\_observation\textquotesingle{}}
\CommentTok{\# and \textquotesingle{}sourceID\textquotesingle{}}
\NormalTok{biotimeDATA[ ,}\StringTok{\textasciigrave{}}\AttributeTok{:=}\StringTok{\textasciigrave{}}\NormalTok{(}\AttributeTok{fullname =}\NormalTok{ species,}
                   \AttributeTok{coordinateUncertaintyInMeters =} \ConstantTok{NA}\NormalTok{,}
                   \AttributeTok{institutionCode =} \ConstantTok{NA}\NormalTok{,}
                   \AttributeTok{countryCode =} \ConstantTok{NA}\NormalTok{,}
                   \AttributeTok{basisOfRecord =} \ConstantTok{NA}\NormalTok{,}
                   \AttributeTok{establishmentMeans =} \ConstantTok{NA}\NormalTok{,}
                   \AttributeTok{is\_cultivated\_observation =} \StringTok{"No"}\NormalTok{,}
                   \AttributeTok{sourceID =} \StringTok{\textquotesingle{}BIOTIME\textquotesingle{}}\NormalTok{)]}

\CommentTok{\# set column order}
\NormalTok{data.table}\SpecialCharTok{::}\FunctionTok{setcolorder}\NormalTok{(biotimeDATA, colNames)}

\CommentTok{\# set the key to the species column to enable fast binary search}
\NormalTok{data.table}\SpecialCharTok{::}\FunctionTok{setkey}\NormalTok{(biotimeDATA, }\StringTok{\textquotesingle{}species\textquotesingle{}}\NormalTok{)}
\end{Highlighting}
\end{Shaded}

\hypertarget{genesys}{%
\subsection{\texorpdfstring{\emph{Genesys}}{Genesys}}\label{genesys}}

\begin{Shaded}
\begin{Highlighting}[]

\CommentTok{\# read in the data}
\NormalTok{genesysDATA }\OtherTok{\textless{}{-}}\NormalTok{ data.table}\SpecialCharTok{::}\FunctionTok{fread}\NormalTok{(}\FunctionTok{paste}\NormalTok{(}\StringTok{"unzip {-}p"}\NormalTok{,}\StringTok{"path/to/file.zip"}\NormalTok{),}
                                     \AttributeTok{header=}\ConstantTok{TRUE}\NormalTok{,}
                                     \AttributeTok{showProgress=}\ConstantTok{FALSE}\NormalTok{,}
                                     \AttributeTok{na.strings=}\FunctionTok{c}\NormalTok{(}\StringTok{""}\NormalTok{,}\ConstantTok{NA}\NormalTok{),}
                                     \AttributeTok{select=} \FunctionTok{c}\NormalTok{(}\StringTok{"GENUS"}\NormalTok{,}
                                    \StringTok{"SPECIES"}\NormalTok{,}
                                    \StringTok{"SUBTAXA"}\NormalTok{,}
                                    \StringTok{"DECLONGITUDE"}\NormalTok{,}
                                    \StringTok{"DECLATITUDE"}\NormalTok{,}
                                    \StringTok{"ORIGCTY"}\NormalTok{,}
                                    \StringTok{"COLLSRC"}\NormalTok{,}
                                    \StringTok{"INSTCODE"}\NormalTok{,}
                                    \StringTok{"UUID"}\NormalTok{,}
                                    \StringTok{"COLLDATE"}\NormalTok{,}
                                    \StringTok{"COORDUNCERT"}\NormalTok{,}
                                    \StringTok{"SAMPSTAT"}\NormalTok{))}

\CommentTok{\#{-}{-}{-}{-}{-}{-}{-}{-}{-}{-}{-}{-}{-}{-}{-}}
\CommentTok{\#= formatting}
\CommentTok{\#{-}{-}{-}{-}{-}{-}{-}{-}{-}{-}{-}{-}{-}{-}{-}}

\CommentTok{\# change the column names}
\NormalTok{data.table}\SpecialCharTok{::}\FunctionTok{setnames}\NormalTok{(genesysDATA,}
                     \FunctionTok{c}\NormalTok{(}\StringTok{"genus"}\NormalTok{,}
                       \StringTok{"species"}\NormalTok{,}
                       \StringTok{"subtaxa"}\NormalTok{,}
                       \StringTok{"decimalLongitude"}\NormalTok{,}
                       \StringTok{"decimalLatitude"}\NormalTok{,}
                       \StringTok{"countryCode"}\NormalTok{,}
                       \StringTok{"basisOfRecord"}\NormalTok{,}
                       \StringTok{"institutionCode"}\NormalTok{,}
                       \StringTok{"gbifID"}\NormalTok{,}
                       \StringTok{"year"}\NormalTok{,}
                       \StringTok{"coordinateUncertaintyInMeters"}\NormalTok{,}
                       \StringTok{"is\_cultivated\_observation"}\NormalTok{))}

\CommentTok{\# format the data: }
\CommentTok{\# concatenate genus+species,}
\CommentTok{\# create full species name,}
\CommentTok{\# extract the year of the collection,}
\CommentTok{\# transform information about cultivated observation and basis of records in categorical variable}
\NormalTok{genesysDATA[, }\StringTok{\textasciigrave{}}\AttributeTok{:=}\StringTok{\textasciigrave{}}\NormalTok{(}\AttributeTok{species =} \FunctionTok{paste}\NormalTok{(genus, species),}
                  \AttributeTok{fullname =} \FunctionTok{ifelse}\NormalTok{(}\FunctionTok{is.na}\NormalTok{(subtaxa),}
                                    \FunctionTok{paste}\NormalTok{(genus, species),}
                                    \FunctionTok{paste}\NormalTok{(genus, species, subtaxa)),}
                  \AttributeTok{year =} \FunctionTok{as.numeric}\NormalTok{(stringr}\SpecialCharTok{::}\FunctionTok{str\_extract}\NormalTok{(year,}\StringTok{"[[:digit:]]\{4\}"}\NormalTok{)),}
                  \AttributeTok{is\_cultivated\_observation =} \FunctionTok{ifelse}\NormalTok{(is\_cultivated\_observation}\SpecialCharTok{==}\DecValTok{999}\NormalTok{,}
                                                     \ConstantTok{NA}\NormalTok{,}
                                                     \FunctionTok{ifelse}\NormalTok{(is\_cultivated\_observation }\SpecialCharTok{\textgreater{}} \DecValTok{300}\NormalTok{,}
                                                            \StringTok{"Yes"}\NormalTok{,}
                                                            \StringTok{"No"}\NormalTok{)),}
                  \AttributeTok{basisOfRecord =} \FunctionTok{cut}\NormalTok{(basisOfRecord, }
                                      \AttributeTok{breaks=}\FunctionTok{c}\NormalTok{(}\DecValTok{10}\NormalTok{, }\DecValTok{16}\NormalTok{, }\DecValTok{29}\NormalTok{, }\DecValTok{59}\NormalTok{, }\DecValTok{63}\NormalTok{),}
                                      \AttributeTok{labels=}\FunctionTok{c}\NormalTok{(}
                                        \StringTok{"Wild habitat"}\NormalTok{,}
                                        \ConstantTok{NA}\NormalTok{,}
                                        \StringTok{"Cultivated habitat"}\NormalTok{,}
                                        \StringTok{"Wild habitat"}\NormalTok{)}
\NormalTok{                                      ))]}

\CommentTok{\# create additional columns: \textquotesingle{}establishmentMeans\textquotesingle{}, \textquotesingle{}individualCount\textquotesingle{} and }
\CommentTok{\# \textquotesingle{}sourceID\textquotesingle{} and delete \textquotesingle{}genus\textquotesingle{} and \textquotesingle{}subtaxa\textquotesingle{}}
\NormalTok{genesysDATA[ ,}\StringTok{\textasciigrave{}}\AttributeTok{:=}\StringTok{\textasciigrave{}}\NormalTok{(}\AttributeTok{genus =} \ConstantTok{NULL}\NormalTok{,}
                  \AttributeTok{subtaxa =} \ConstantTok{NULL}\NormalTok{,}
                  \AttributeTok{establishmentMeans =} \ConstantTok{NA}\NormalTok{,}
                  \AttributeTok{individualCount =} \ConstantTok{NA}\NormalTok{,}
                  \AttributeTok{sourceID =} \StringTok{\textquotesingle{}GENESYS\textquotesingle{}}\NormalTok{)]}

\CommentTok{\# set column order}
\NormalTok{data.table}\SpecialCharTok{::}\FunctionTok{setcolorder}\NormalTok{(genesysDATA, colNames)}

\CommentTok{\# set the key to the species column to enable fast binary search}
\NormalTok{data.table}\SpecialCharTok{::}\FunctionTok{setkey}\NormalTok{(genesysDATA, }\StringTok{\textquotesingle{}species\textquotesingle{}}\NormalTok{)}
\end{Highlighting}
\end{Shaded}

\hypertarget{references}{%
\section*{References}\label{references}}
\addcontentsline{toc}{section}{References}

\hypertarget{refs}{}
\begin{CSLReferences}{1}{0}
\leavevmode\vadjust pre{\hypertarget{ref-rwcvp}{}}%
Brown, Matilda J. M., Barnaby E. Walker, Nicholas Black, Rafaël
Govaerts, Ian Ondo, Robert Turner, and Eimear Nic Lughadha. 2023.
{``rWCVP: A Companion r Package to the World Checklist of Vascular
Plants.''} \emph{New Phytologist}.

\leavevmode\vadjust pre{\hypertarget{ref-rgbif}{}}%
Chamberlain, Scott, and Carl Boettiger. 2017. {``R Python, and Ruby
Clients for GBIF Species Occurrence Data.''} \emph{PeerJ PrePrints}.
\url{https://doi.org/10.7287/peerj.preprints.3304v1}.

\leavevmode\vadjust pre{\hypertarget{ref-data_table}{}}%
Dowle, Matt, and Arun Srinivasan. 2021. \emph{Data.table: Extension of
`Data.frame`}. \url{https://CRAN.R-project.org/package=data.table}.

\leavevmode\vadjust pre{\hypertarget{ref-BIEN}{}}%
Maitner, Brian. 2020. \emph{BIEN: Tools for Accessing the Botanical
Information and Ecology Network Database}.
\url{https://CRAN.R-project.org/package=BIEN}.

\leavevmode\vadjust pre{\hypertarget{ref-taxize}{}}%
Scott Chamberlain, and Eduard Szocs. 2013. {``Taxize - Taxonomic Search
and Retrieval in r.''} \emph{F1000Research}.
\url{https://f1000research.com/articles/2-191/v2}.

\end{CSLReferences}

\end{document}
