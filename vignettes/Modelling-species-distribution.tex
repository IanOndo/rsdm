% Options for packages loaded elsewhere
\PassOptionsToPackage{unicode}{hyperref}
\PassOptionsToPackage{hyphens}{url}
%
\documentclass[
]{article}
\usepackage{amsmath,amssymb}
\usepackage{lmodern}
\usepackage{iftex}
\ifPDFTeX
  \usepackage[T1]{fontenc}
  \usepackage[utf8]{inputenc}
  \usepackage{textcomp} % provide euro and other symbols
\else % if luatex or xetex
  \usepackage{unicode-math}
  \defaultfontfeatures{Scale=MatchLowercase}
  \defaultfontfeatures[\rmfamily]{Ligatures=TeX,Scale=1}
\fi
% Use upquote if available, for straight quotes in verbatim environments
\IfFileExists{upquote.sty}{\usepackage{upquote}}{}
\IfFileExists{microtype.sty}{% use microtype if available
  \usepackage[]{microtype}
  \UseMicrotypeSet[protrusion]{basicmath} % disable protrusion for tt fonts
}{}
\makeatletter
\@ifundefined{KOMAClassName}{% if non-KOMA class
  \IfFileExists{parskip.sty}{%
    \usepackage{parskip}
  }{% else
    \setlength{\parindent}{0pt}
    \setlength{\parskip}{6pt plus 2pt minus 1pt}}
}{% if KOMA class
  \KOMAoptions{parskip=half}}
\makeatother
\usepackage{xcolor}
\usepackage[margin=1in]{geometry}
\usepackage{color}
\usepackage{fancyvrb}
\newcommand{\VerbBar}{|}
\newcommand{\VERB}{\Verb[commandchars=\\\{\}]}
\DefineVerbatimEnvironment{Highlighting}{Verbatim}{commandchars=\\\{\}}
% Add ',fontsize=\small' for more characters per line
\usepackage{framed}
\definecolor{shadecolor}{RGB}{248,248,248}
\newenvironment{Shaded}{\begin{snugshade}}{\end{snugshade}}
\newcommand{\AlertTok}[1]{\textcolor[rgb]{0.94,0.16,0.16}{#1}}
\newcommand{\AnnotationTok}[1]{\textcolor[rgb]{0.56,0.35,0.01}{\textbf{\textit{#1}}}}
\newcommand{\AttributeTok}[1]{\textcolor[rgb]{0.77,0.63,0.00}{#1}}
\newcommand{\BaseNTok}[1]{\textcolor[rgb]{0.00,0.00,0.81}{#1}}
\newcommand{\BuiltInTok}[1]{#1}
\newcommand{\CharTok}[1]{\textcolor[rgb]{0.31,0.60,0.02}{#1}}
\newcommand{\CommentTok}[1]{\textcolor[rgb]{0.56,0.35,0.01}{\textit{#1}}}
\newcommand{\CommentVarTok}[1]{\textcolor[rgb]{0.56,0.35,0.01}{\textbf{\textit{#1}}}}
\newcommand{\ConstantTok}[1]{\textcolor[rgb]{0.00,0.00,0.00}{#1}}
\newcommand{\ControlFlowTok}[1]{\textcolor[rgb]{0.13,0.29,0.53}{\textbf{#1}}}
\newcommand{\DataTypeTok}[1]{\textcolor[rgb]{0.13,0.29,0.53}{#1}}
\newcommand{\DecValTok}[1]{\textcolor[rgb]{0.00,0.00,0.81}{#1}}
\newcommand{\DocumentationTok}[1]{\textcolor[rgb]{0.56,0.35,0.01}{\textbf{\textit{#1}}}}
\newcommand{\ErrorTok}[1]{\textcolor[rgb]{0.64,0.00,0.00}{\textbf{#1}}}
\newcommand{\ExtensionTok}[1]{#1}
\newcommand{\FloatTok}[1]{\textcolor[rgb]{0.00,0.00,0.81}{#1}}
\newcommand{\FunctionTok}[1]{\textcolor[rgb]{0.00,0.00,0.00}{#1}}
\newcommand{\ImportTok}[1]{#1}
\newcommand{\InformationTok}[1]{\textcolor[rgb]{0.56,0.35,0.01}{\textbf{\textit{#1}}}}
\newcommand{\KeywordTok}[1]{\textcolor[rgb]{0.13,0.29,0.53}{\textbf{#1}}}
\newcommand{\NormalTok}[1]{#1}
\newcommand{\OperatorTok}[1]{\textcolor[rgb]{0.81,0.36,0.00}{\textbf{#1}}}
\newcommand{\OtherTok}[1]{\textcolor[rgb]{0.56,0.35,0.01}{#1}}
\newcommand{\PreprocessorTok}[1]{\textcolor[rgb]{0.56,0.35,0.01}{\textit{#1}}}
\newcommand{\RegionMarkerTok}[1]{#1}
\newcommand{\SpecialCharTok}[1]{\textcolor[rgb]{0.00,0.00,0.00}{#1}}
\newcommand{\SpecialStringTok}[1]{\textcolor[rgb]{0.31,0.60,0.02}{#1}}
\newcommand{\StringTok}[1]{\textcolor[rgb]{0.31,0.60,0.02}{#1}}
\newcommand{\VariableTok}[1]{\textcolor[rgb]{0.00,0.00,0.00}{#1}}
\newcommand{\VerbatimStringTok}[1]{\textcolor[rgb]{0.31,0.60,0.02}{#1}}
\newcommand{\WarningTok}[1]{\textcolor[rgb]{0.56,0.35,0.01}{\textbf{\textit{#1}}}}
\usepackage{graphicx}
\makeatletter
\def\maxwidth{\ifdim\Gin@nat@width>\linewidth\linewidth\else\Gin@nat@width\fi}
\def\maxheight{\ifdim\Gin@nat@height>\textheight\textheight\else\Gin@nat@height\fi}
\makeatother
% Scale images if necessary, so that they will not overflow the page
% margins by default, and it is still possible to overwrite the defaults
% using explicit options in \includegraphics[width, height, ...]{}
\setkeys{Gin}{width=\maxwidth,height=\maxheight,keepaspectratio}
% Set default figure placement to htbp
\makeatletter
\def\fps@figure{htbp}
\makeatother
\setlength{\emergencystretch}{3em} % prevent overfull lines
\providecommand{\tightlist}{%
  \setlength{\itemsep}{0pt}\setlength{\parskip}{0pt}}
\setcounter{secnumdepth}{-\maxdimen} % remove section numbering
\newlength{\cslhangindent}
\setlength{\cslhangindent}{1.5em}
\newlength{\csllabelwidth}
\setlength{\csllabelwidth}{3em}
\newlength{\cslentryspacingunit} % times entry-spacing
\setlength{\cslentryspacingunit}{\parskip}
\newenvironment{CSLReferences}[2] % #1 hanging-ident, #2 entry spacing
 {% don't indent paragraphs
  \setlength{\parindent}{0pt}
  % turn on hanging indent if param 1 is 1
  \ifodd #1
  \let\oldpar\par
  \def\par{\hangindent=\cslhangindent\oldpar}
  \fi
  % set entry spacing
  \setlength{\parskip}{#2\cslentryspacingunit}
 }%
 {}
\usepackage{calc}
\newcommand{\CSLBlock}[1]{#1\hfill\break}
\newcommand{\CSLLeftMargin}[1]{\parbox[t]{\csllabelwidth}{#1}}
\newcommand{\CSLRightInline}[1]{\parbox[t]{\linewidth - \csllabelwidth}{#1}\break}
\newcommand{\CSLIndent}[1]{\hspace{\cslhangindent}#1}

\usepackage{tcolorbox}

\definecolor{mybluebackground}{HTML}{d9edf7}
\definecolor{mybluecolor}{HTML}{31708f}
\definecolor{myblueborder}{HTML}{bce8f1}
\definecolor{mybrownbackground}{HTML}{fcf8e3}
\definecolor{mybrowncolor}{HTML}{8a6d3b}
\definecolor{mybrownborder}{HTML}{faebcc}

\newtcolorbox{infobox}{
  colback=mybluebackground,
  colframe=myblueborder,
  coltext=mybluecolor,
  boxsep=3pt,
  boxrule=2pt,
  arc=4pt}

\newtcolorbox{warningbox}{
  colback=mybrownbackground,
  colframe=mybrownborder,
  coltext=mybrowncolor,
  boxsep=3pt,
  boxrule=2pt,
  arc=4pt}

\usepackage{amsmath}
\usepackage{booktabs}
\usepackage{float}
\usepackage{subcaption}
\usepackage{caption}
\floatplacement{figure}{H}
\ifLuaTeX
  \usepackage{selnolig}  % disable illegal ligatures
\fi
\IfFileExists{bookmark.sty}{\usepackage{bookmark}}{\usepackage{hyperref}}
\IfFileExists{xurl.sty}{\usepackage{xurl}}{} % add URL line breaks if available
\urlstyle{same} % disable monospaced font for URLs
\hypersetup{
  pdftitle={Modelling species distribution},
  pdfauthor={Ian Ondo},
  hidelinks,
  pdfcreator={LaTeX via pandoc}}

\title{Modelling species distribution}
\author{Ian Ondo}
\date{2023-05-22}

\begin{document}
\maketitle

{
\setcounter{tocdepth}{2}
\tableofcontents
}
\begin{warningbox}

\textbf{Package notes:}\\
We need to install the following set of R packages to successfully run
the codes in this vignette:

\begin{itemize}
\tightlist
\item
  \textbf{geodata} (Hijmans et al. 2023)
\item
  \textbf{dismo} (Hijmans et al. 2021)
\item
  \textbf{usdm} (Naimi et al. 2014)
\item
  \textbf{ade4} (Dray, Dufour, and Chessel 2007)
\item
  \textbf{ggplot2} (Wickham 2016)
\item
  \textbf{kableExtra} (Zhu 2023)
\item
  \textbf{dplyr} (Wickham et al. 2023)
\end{itemize}

\end{warningbox}

\hypertarget{introduction}{%
\section{Introduction}\label{introduction}}

In this vignette, we will focus on modelling the distribution of
\emph{Abies spectabilis} (hereafter \emph{A. spectabilis}) with the
\href{https://github.com/IanOndo/UsefulPlants}{\textbf{\textcolor{black}{\underline{UsefulPlants}}}}
package. We will walk through the modelling process and introduce key
functions and R codes used to assist each individual steps.

\hypertarget{i.-data-preparation}{%
\section{I. Data preparation}\label{i.-data-preparation}}

Our occurrence dataset contains 40 rows of occurrence locations of
\emph{A. spectabilis} curated beforehand (see vignette
\href{}{\textbf{\textcolor{black}{\underline{Data gathering and pre-processing}}}}.
The first few rows look like this:

\begin{table}[H]

\caption{\label{tab:head-occurrence-table}Occurrence dataset of \textit{A. spectabilis}}
\centering
\resizebox{\linewidth}{!}{
\begin{tabular}[t]{lcclcc}
\toprule
species & decimalLongitude & decimalLatitude & year & countryCode & basisOfRecord\\
\midrule
Abies spectabilis & 88.45 & 27.55 & ** & IND & PRESERVED\_SPECIMEN\\
Abies spectabilis & 88.85 & 27.47 & ** & CHN & PRESERVED\_SPECIMEN\\
Abies spectabilis & 85.29 & 28.85 & 1975 & CHN & PRESERVED\_SPECIMEN\\
Abies spectabilis & 88.90 & 27.48 & 1975 & CHN & PRESERVED\_SPECIMEN\\
Abies spectabilis & 87.12 & 28.65 & 1959 & CHN & PRESERVED\_SPECIMEN\\
\addlinespace
Abies spectabilis & 87.76 & 28.36 & 1975 & CHN & PRESERVED\_SPECIMEN\\
\bottomrule
\multicolumn{6}{l}{\rule{0pt}{1em}\textit{Note: }}\\
\multicolumn{6}{l}{\rule{0pt}{1em}Only relevant columns are shown}\\
\end{tabular}}
\end{table}

For simplicity, we are going to use the R package \textbf{geodata} to
obtain a set of environmental layers. Various climate, elevation and
soil-related raster datasets can be directly downloaded from R with this
package.

\begin{Shaded}
\begin{Highlighting}[]
\FunctionTok{library}\NormalTok{(geodata)}

\CommentTok{\# setup a (temporary) directory to store your environmental raster layers}
\NormalTok{my\_env\_tmp\_dir }\OtherTok{=} \FunctionTok{tempdir}\NormalTok{()}

\CommentTok{\# spatial resolution}
\NormalTok{my\_res }\OtherTok{=} \DecValTok{10} \CommentTok{\# 10 minutes degree resolution}

\CommentTok{\# download global dataset of WorldClim bioclimatic variables }
\NormalTok{wc }\OtherTok{=}\NormalTok{ geodata}\SpecialCharTok{::}\FunctionTok{worldclim\_global}\NormalTok{(}\AttributeTok{var=}\StringTok{"bio"}\NormalTok{,}
                               \AttributeTok{res=}\NormalTok{my\_res,}
                               \AttributeTok{path =}\NormalTok{ my\_env\_tmp\_dir)}

\CommentTok{\# download global dataset of SRTM elevation model}
\NormalTok{alt }\OtherTok{=}\NormalTok{ geodata}\SpecialCharTok{::}\FunctionTok{elevation\_global}\NormalTok{(}\AttributeTok{res=}\NormalTok{my\_res,}
                                \AttributeTok{path =}\NormalTok{ my\_env\_tmp\_dir)}

\CommentTok{\# download the human footprint index}
\NormalTok{hfp }\OtherTok{=}\NormalTok{ geodata}\SpecialCharTok{::}\FunctionTok{footprint}\NormalTok{(}\AttributeTok{year =} \DecValTok{2009}\NormalTok{,}
                   \AttributeTok{path =}\NormalTok{ my\_env\_tmp\_dir)}

\CommentTok{\# uncomment below to download soil{-}related variable, but be mindful,}
\CommentTok{\# global downloads at 30 seconds resolution take a while !}
\CommentTok{\#}
\CommentTok{\# download global dataset of soil organic carbon and pH at 30{-}60cm depth}
\CommentTok{\# soc = geodata::soil\_world(var="soc",depth=60, path=my\_env\_tmp\_dir)}
\CommentTok{\# pH = geodata::soil\_world(var="phh2o", depth=60, path=my\_env\_tmp\_dir)}
\end{Highlighting}
\end{Shaded}

\hypertarget{defining-the-training-area}{%
\subsection{\texorpdfstring{\emph{Defining the training
area}}{Defining the training area}}\label{defining-the-training-area}}

We need to delineate the geographic area in which we will train our
model. This training area should ideally represent the environmental
conditions available or accessible to the species. In
\textbf{UsefulPlants}, it extends to the boundaries of the biomes or
ecoregions where species' occurrence locations were found. We are going
to use the function \texttt{make\_geographic\_domain} which relies on
\emph{Terrestrial Ecoregions Of the World} (TEOW) and some user-defined
settings to create the training area from our occurrence data points.

\begin{Shaded}
\begin{Highlighting}[]

\FunctionTok{library}\NormalTok{(UsefulPlants)}

\CommentTok{\# create your training area}
\NormalTok{my\_training\_area }\OtherTok{\textless{}{-}} \FunctionTok{make\_geographic\_domain}\NormalTok{(}
  
  \CommentTok{\# path to the occurrence records file}
  \AttributeTok{loc\_dat=}\NormalTok{params}\SpecialCharTok{$}\NormalTok{occ\_file, }
  
  \CommentTok{\# optional vector of longitude/latitude}
  \AttributeTok{coordHeaders =} \FunctionTok{c}\NormalTok{(}\StringTok{"decimalLongitude"}\NormalTok{,}\StringTok{"decimalLatitude"}\NormalTok{), }
  
  \CommentTok{\# build alpha{-}hull from the set of points}
  \AttributeTok{do.alpha\_hull =} \ConstantTok{TRUE}\NormalTok{, }
  
  \CommentTok{\# do not dissolve borders between polygons (i.e. biomes or ecoregions)}
  \AttributeTok{dissolve =} \ConstantTok{FALSE}\NormalTok{,}
  
  \CommentTok{\# build the alpha{-}hull with at least 95\% of the points}
  \AttributeTok{fraction =} \FloatTok{0.95}\NormalTok{, }
  
  \CommentTok{\# get biomes whose intersection area with the alpha{-}hull \textgreater{}= 10\% (of their area),}
  \CommentTok{\# otherwise get ecoregions}
  \AttributeTok{min\_area\_cover =} \FloatTok{0.1}\NormalTok{, }
  
  \CommentTok{\# get biomes with \textgreater{}= 10 points inside, otherwise get ecoregions}
  \AttributeTok{min\_occ\_number =} \DecValTok{10}\NormalTok{,}
  
  \CommentTok{\# optional terrestrial land vector maps}
  \AttributeTok{land\_file =}\NormalTok{ RGeodata}\SpecialCharTok{::}\NormalTok{terrestrial\_lands, }
  
  \CommentTok{\# run quietly. set to TRUE to display stepwise detailed information}
  \AttributeTok{verbose=}\ConstantTok{FALSE} 
\NormalTok{)}
\end{Highlighting}
\end{Shaded}

\begin{infobox}

Internally, \texttt{make\_geographic\_domain}:

\begin{itemize}
\item
  Built an alpha-hull around occurrence points (include at least 95\% of
  points by default)
\item
  Detected biomes in the TEOW dataset that intersect with the alpha-hull
\item
  Made sure that these biomes include \textgreater= N points (default is
  N=10) and that the intersection area with the alpha-hull \textgreater=
  X\% (default is X=10\% of the biome area)
\item
  Made sure to include ecoregions of points excluded by the alpha-hull,
  or geographic outliers if they exist.
\end{itemize}

\end{infobox}

Let's have a look at the geographic region selected to be sure that it
makes sense by overlaying our occurrence records on the map.

\begin{Shaded}
\begin{Highlighting}[]
 
  \CommentTok{\# plot the region selected and overlay your occurrence data points}
\NormalTok{  mplt }\OtherTok{\textless{}{-}}\NormalTok{ my\_training\_area }\SpecialCharTok{\%\textgreater{}\%}\NormalTok{ ggplot2}\SpecialCharTok{::}\FunctionTok{ggplot}\NormalTok{() }\SpecialCharTok{+}
\NormalTok{    ggplot2}\SpecialCharTok{::}\FunctionTok{geom\_sf}\NormalTok{(ggplot2}\SpecialCharTok{::}\FunctionTok{aes}\NormalTok{(}\AttributeTok{fill =}\NormalTok{ REGION\_NAME)) }\SpecialCharTok{+}
\NormalTok{    ggplot2}\SpecialCharTok{::}\FunctionTok{geom\_point}\NormalTok{(}\AttributeTok{data =}\NormalTok{ occ\_data, }
\NormalTok{                        ggplot2}\SpecialCharTok{::}\FunctionTok{aes}\NormalTok{(}\AttributeTok{x =}\NormalTok{ decimalLongitude, }\AttributeTok{y =}\NormalTok{ decimalLatitude),}
                        \AttributeTok{size =} \DecValTok{2}\NormalTok{,                                                }
                        \AttributeTok{shape =} \DecValTok{4}\NormalTok{,                                              }
                        \AttributeTok{fill =} \StringTok{"black"}\NormalTok{) }\SpecialCharTok{+}                                       
\NormalTok{    ggplot2}\SpecialCharTok{::}\FunctionTok{labs}\NormalTok{(}\AttributeTok{fill =} \StringTok{"Biomes/Ecoregions"}\NormalTok{,) }\SpecialCharTok{+} 
\NormalTok{    ggplot2}\SpecialCharTok{::}\FunctionTok{theme\_bw}\NormalTok{() }\SpecialCharTok{+} 
\NormalTok{    ggplot2}\SpecialCharTok{::}\FunctionTok{theme}\NormalTok{(}
    \AttributeTok{text=}\NormalTok{ggplot2}\SpecialCharTok{::}\FunctionTok{element\_text}\NormalTok{(}\AttributeTok{family=}\StringTok{"serif"}\NormalTok{),}
    \AttributeTok{legend.title =}\NormalTok{ ggplot2}\SpecialCharTok{::}\FunctionTok{element\_text}\NormalTok{(}\AttributeTok{size=}\DecValTok{7}\NormalTok{, }\AttributeTok{face =} \StringTok{"bold"}\NormalTok{),}
    \AttributeTok{legend.key.size =} \FunctionTok{unit}\NormalTok{(}\FloatTok{0.4}\NormalTok{,}\StringTok{\textquotesingle{}cm\textquotesingle{}}\NormalTok{),}
    \AttributeTok{legend.text =}\NormalTok{ ggplot2}\SpecialCharTok{::}\FunctionTok{element\_text}\NormalTok{(}\AttributeTok{size=}\DecValTok{5}\NormalTok{),}
    \AttributeTok{axis.text=}\NormalTok{ggplot2}\SpecialCharTok{::}\FunctionTok{element\_text}\NormalTok{(}\AttributeTok{colour=}\StringTok{"black"}\NormalTok{, }\AttributeTok{size=}\DecValTok{6}\NormalTok{),}
    \AttributeTok{axis.title=}\NormalTok{ggplot2}\SpecialCharTok{::}\FunctionTok{element\_text}\NormalTok{(}\AttributeTok{colour=}\StringTok{"black"}\NormalTok{, }\AttributeTok{size=}\DecValTok{8}\NormalTok{,}\AttributeTok{face=}\StringTok{"bold"}\NormalTok{)}
\NormalTok{    ) }\SpecialCharTok{+}
\NormalTok{    ggplot2}\SpecialCharTok{::}\FunctionTok{xlab}\NormalTok{(}\StringTok{"Longitude"}\NormalTok{) }\SpecialCharTok{+}\NormalTok{ ggplot2}\SpecialCharTok{::}\FunctionTok{ylab}\NormalTok{(}\StringTok{"Latitude"}\NormalTok{) }\SpecialCharTok{+} 
\NormalTok{    ggplot2}\SpecialCharTok{::}\FunctionTok{scale\_fill\_hue}\NormalTok{(}\AttributeTok{l=}\DecValTok{40}\NormalTok{)}
    
\NormalTok{mplt}
\end{Highlighting}
\end{Shaded}

\begin{warningbox}
For this vignette, we intentionally displayed the biomes and ecoregions
composing the training area of our species by setting
\textbf{\texttt{dissolve=FALSE}}, however, for the rest of the analysis
make sure to set \textbf{\texttt{dissolve=TRUE}} when creating your
training area to merge all biomes and ecoregions into one geographic
region.

\end{warningbox}

The region selected seems okay, but if not satisfied with it, try to
change some parameters e.g.~\texttt{fraction}, \texttt{min\_area\_cover}
or \texttt{min\_occ\_number} until you get the desired training area.

\hypertarget{selecting-environmental-predictors}{%
\subsection{\texorpdfstring{\emph{Selecting environmental
predictors}}{Selecting environmental predictors}}\label{selecting-environmental-predictors}}

First of all, the environmental rasters were downloaded at different
spatial grain, extent or projection, so to
\underline{ensure that all datasets spatially match}, we need to:

\begin{itemize}
\item
  \textbf{clip} all layers to the training area previously defined to
  speed up the computations.
\item
  \textbf{resample} all layers using the geographic information of a
  raster taken as reference for the study area.
\end{itemize}

We use the function \texttt{maskCover} to clip the rasters. It does a
better job to approximate the irregular shape of the training area by
using the intersection of the entire grid cell with the training area
polygon rather than the grid cell's centroid like other functions.

\begin{Shaded}
\begin{Highlighting}[]

\CommentTok{\# clip climate data}
\NormalTok{wc\_clipped }\OtherTok{=} \FunctionTok{maskCover}\NormalTok{(wc, my\_training\_area)}

\CommentTok{\# clip elevation data}
\NormalTok{alt\_clipped }\OtherTok{=} \FunctionTok{maskCover}\NormalTok{(alt, my\_training\_area)}

\CommentTok{\# clip human footprint}
\NormalTok{hfp\_clipped }\OtherTok{=} \FunctionTok{maskCover}\NormalTok{(hfp, my\_training\_area)}

\CommentTok{\# resample elevation and human footprint layers using a bioclimatic layer as a reference raster}
\NormalTok{alt\_resampled }\OtherTok{\textless{}{-}}\NormalTok{ raster}\SpecialCharTok{::}\FunctionTok{resample}\NormalTok{(}
\NormalTok{  alt\_clipped,}
\NormalTok{  wc\_clipped[[}\DecValTok{1}\NormalTok{]],}
  \AttributeTok{method =} \StringTok{"ngb"}  \CommentTok{\# keep discrete values}
\NormalTok{)}

\NormalTok{hfp\_resampled }\OtherTok{\textless{}{-}}\NormalTok{ raster}\SpecialCharTok{::}\FunctionTok{resample}\NormalTok{(}
\NormalTok{  hfp\_clipped,}
\NormalTok{  wc\_clipped[[}\DecValTok{1}\NormalTok{]]}
\NormalTok{)}
\end{Highlighting}
\end{Shaded}

Now, let's compute three more topography-related variables, the
\emph{land slope (S)}, the \emph{Topographic Ruggedness Index (TRI)} and
the \emph{terrain roughness (R)}

\begin{Shaded}
\begin{Highlighting}[]

\CommentTok{\# compute S, TRI and R}
\NormalTok{topo }\OtherTok{\textless{}{-}}\NormalTok{ raster}\SpecialCharTok{::}\FunctionTok{terrain}\NormalTok{(alt\_resampled, }\AttributeTok{opt=}\FunctionTok{c}\NormalTok{(}\StringTok{"slope"}\NormalTok{,}\StringTok{"TRI"}\NormalTok{,}\StringTok{"roughness"}\NormalTok{))}

\CommentTok{\# add to climate and human footprint layers}
\NormalTok{env\_data }\OtherTok{\textless{}{-}}\NormalTok{ raster}\SpecialCharTok{::}\FunctionTok{stack}\NormalTok{(wc\_clipped, hfp\_resampled, topo)}
\end{Highlighting}
\end{Shaded}

Variable selection is not a linear process, one needs to balance
statistical evidence with ecological meaning, and this usually entails
going back and forth between different steps until finding a convincing
set of variables.

\begin{infobox}

A good practice for selecting environmental predictors in plant studies,
starts by gathering a large set of factors known to influence the
physiological response of plants to environmental conditions such as
factors related to nutrition (availability of water and nutrients) and
energy (availability of light). Then, progressively removing redundant
information (i.e.~factors highly correlated) while keeping:

\begin{itemize}
\item
  direct factors over proxies
\item
  factors whose effects are easier to interpret
\item
  annual trend and seasonal effects over monthly extremes (for large
  scale studies)
\item
  limiting factors and stress effects over maxima.
\end{itemize}

\end{infobox}

There are many ways to do a variable selection, but here is one example,
of what a variable selection may look like.

\hypertarget{pca}{%
\subsubsection{\texorpdfstring{\emph{PCA}}{PCA}}\label{pca}}

We can start our selection by performing a Principal component Analysis
(PCA) to identify variables that most drive the environmental conditions
of our study area (variables contributing the most to the first axis of
the PCA). This will give a first broad idea of which variables best
describe our training area.

\begin{Shaded}
\begin{Highlighting}[]

\CommentTok{\# keep the first two principal components (PC)}
\NormalTok{ndim }\OtherTok{=} \DecValTok{2}

\CommentTok{\# perform PCA}
\NormalTok{pca }\OtherTok{\textless{}{-}}\NormalTok{ ade4}\SpecialCharTok{::}\FunctionTok{dudi.pca}\NormalTok{(}\AttributeTok{df=}\NormalTok{env\_data[],}
                      \AttributeTok{center=} \ConstantTok{TRUE}\NormalTok{, }\CommentTok{\# (optional) remove the mean}
                      \AttributeTok{scale=}\ConstantTok{TRUE}\NormalTok{,  }
                      \AttributeTok{scannf=}\NormalTok{F, }\CommentTok{\# do not plot the screeplot}
                      \AttributeTok{nf=}\NormalTok{ndim) }\CommentTok{\# nf is the number of dimensions}

\CommentTok{\# Compute inertia explained by each of dimension}
\NormalTok{inertia }\OtherTok{=}\NormalTok{ pca}\SpecialCharTok{$}\NormalTok{eig}\SpecialCharTok{/}\FunctionTok{sum}\NormalTok{(pca}\SpecialCharTok{$}\NormalTok{eig)}\SpecialCharTok{*}\DecValTok{100}
\NormalTok{m }\OtherTok{\textless{}{-}} \FunctionTok{data.frame}\NormalTok{(}\AttributeTok{Comp=}\FunctionTok{paste0}\NormalTok{(}\StringTok{\textquotesingle{}Dim\textquotesingle{}}\NormalTok{,}\DecValTok{1}\SpecialCharTok{:}\FunctionTok{length}\NormalTok{(inertia)), }\AttributeTok{inertia=}\NormalTok{inertia)}

\CommentTok{\# plot eigenvalues}
\NormalTok{bplt }\OtherTok{\textless{}{-}}\NormalTok{ ggplot2}\SpecialCharTok{::}\FunctionTok{ggplot}\NormalTok{(}\AttributeTok{data=}\NormalTok{m, ggplot2}\SpecialCharTok{::}\FunctionTok{aes}\NormalTok{(}\AttributeTok{x=}\NormalTok{Comp, }\AttributeTok{y=}\NormalTok{inertia)) }\SpecialCharTok{+}
\NormalTok{  ggplot2}\SpecialCharTok{::}\FunctionTok{geom\_bar}\NormalTok{(}\AttributeTok{stat =}\StringTok{"identity"}\NormalTok{, ggplot2}\SpecialCharTok{::}\FunctionTok{aes}\NormalTok{(}\AttributeTok{fill=}\FunctionTok{as.factor}\NormalTok{(inertia)),}
                    \AttributeTok{show.legend =} \ConstantTok{FALSE}\NormalTok{) }\SpecialCharTok{+}
\NormalTok{  ggplot2}\SpecialCharTok{::}\FunctionTok{geom\_text}\NormalTok{(ggplot2}\SpecialCharTok{::}\FunctionTok{aes}\NormalTok{(}\AttributeTok{label=}\FunctionTok{paste0}\NormalTok{(}\FunctionTok{round}\NormalTok{(inertia,}\DecValTok{2}\NormalTok{),}\StringTok{"\%"}\NormalTok{),}
                                  \AttributeTok{y=}\NormalTok{inertia}\FloatTok{+0.1}\NormalTok{, }\AttributeTok{fontface=}\StringTok{\textquotesingle{}bold\textquotesingle{}}\NormalTok{),}
                                  \AttributeTok{vjust=}\DecValTok{0}\NormalTok{,}
                                  \AttributeTok{color=}\StringTok{"black"}\NormalTok{,}
                                  \AttributeTok{position =}\NormalTok{ ggplot2}\SpecialCharTok{::}\FunctionTok{position\_dodge}\NormalTok{(}\DecValTok{1}\NormalTok{), }\AttributeTok{size=}\FloatTok{4.5}\NormalTok{) }\SpecialCharTok{+}
\NormalTok{  ggplot2}\SpecialCharTok{::}\FunctionTok{theme\_classic}\NormalTok{() }\SpecialCharTok{+}
\NormalTok{  ggplot2}\SpecialCharTok{::}\FunctionTok{labs}\NormalTok{(}\AttributeTok{title=}\StringTok{"Barplot of Eigenvalues"}\NormalTok{,}
                \AttributeTok{subtitle=}\FunctionTok{paste0}\NormalTok{(}\StringTok{"Using the first "}\NormalTok{,ndim,}\StringTok{" dimensions"}\NormalTok{),}
                \AttributeTok{caption=}\FunctionTok{paste0}\NormalTok{(}\StringTok{"The two first coordinates (dimensions) explain "}\NormalTok{,}
                               \FunctionTok{round}\NormalTok{(}\FunctionTok{sum}\NormalTok{(inertia[}\DecValTok{1}\SpecialCharTok{:}\DecValTok{2}\NormalTok{]),}\DecValTok{2}\NormalTok{),}\StringTok{"\% of the variability"}\NormalTok{)) }\SpecialCharTok{+}
\NormalTok{  ggplot2}\SpecialCharTok{::}\FunctionTok{theme}\NormalTok{(}
    \AttributeTok{text=}\NormalTok{ggplot2}\SpecialCharTok{::}\FunctionTok{element\_text}\NormalTok{(}\AttributeTok{family=}\StringTok{"serif"}\NormalTok{),}
    \AttributeTok{plot.title =}\NormalTok{ ggplot2}\SpecialCharTok{::}\FunctionTok{element\_text}\NormalTok{(}\AttributeTok{face=}\StringTok{"bold"}\NormalTok{, }\AttributeTok{size=}\DecValTok{14}\NormalTok{, }\AttributeTok{hjust =} \DecValTok{0}\NormalTok{),}
    \AttributeTok{axis.text=}\NormalTok{ggplot2}\SpecialCharTok{::}\FunctionTok{element\_text}\NormalTok{(}\AttributeTok{colour=}\StringTok{"black"}\NormalTok{, }\AttributeTok{size=}\DecValTok{12}\NormalTok{),}
    \AttributeTok{axis.title=}\NormalTok{ggplot2}\SpecialCharTok{::}\FunctionTok{element\_text}\NormalTok{(}\AttributeTok{colour=}\StringTok{"black"}\NormalTok{, }\AttributeTok{size=}\DecValTok{16}\NormalTok{,}\AttributeTok{face=}\StringTok{"bold"}\NormalTok{)}
\NormalTok{  ) }\SpecialCharTok{+}
\NormalTok{  ggplot2}\SpecialCharTok{::}\FunctionTok{xlab}\NormalTok{(}\StringTok{"Dimensions"}\NormalTok{) }\SpecialCharTok{+}\NormalTok{ ggplot2}\SpecialCharTok{::}\FunctionTok{ylab}\NormalTok{(}\StringTok{"Percentage of explained variances"}\NormalTok{) }\SpecialCharTok{+}
\NormalTok{  ggplot2}\SpecialCharTok{::}\FunctionTok{scale\_fill\_brewer}\NormalTok{(}\AttributeTok{palette=}\StringTok{\textquotesingle{}Blues\textquotesingle{}}\NormalTok{) }\SpecialCharTok{+} 
\NormalTok{  ggplot2}\SpecialCharTok{::}\FunctionTok{scale\_x\_discrete}\NormalTok{(}\AttributeTok{limits=}\FunctionTok{paste0}\NormalTok{(}\StringTok{\textquotesingle{}Dim\textquotesingle{}}\NormalTok{,}\DecValTok{1}\SpecialCharTok{:}\NormalTok{ndim))}

\NormalTok{bplt}
\end{Highlighting}
\end{Shaded}

\begin{Shaded}
\begin{Highlighting}[]
\CommentTok{\# Evaluation of the absolute contribution of a variable to an axis}
\NormalTok{cont}\OtherTok{=}\NormalTok{ade4}\SpecialCharTok{::}\FunctionTok{inertia.dudi}\NormalTok{(pca,}\AttributeTok{col.inertia=}\ConstantTok{TRUE}\NormalTok{)}\SpecialCharTok{$}\NormalTok{col.abs}

\CommentTok{\# Sort in decreasing order}
\NormalTok{ctr}\OtherTok{=}\NormalTok{cont[}\FunctionTok{order}\NormalTok{(cont[,}\DecValTok{1}\NormalTok{],}\AttributeTok{decreasing=}\ConstantTok{FALSE}\NormalTok{),]}

\DocumentationTok{\#\# Representation of the contribution of each variables to an axis x}
\NormalTok{bplt\_axis\_1 }\OtherTok{\textless{}{-}}\NormalTok{ ggplot2}\SpecialCharTok{::}\FunctionTok{ggplot}\NormalTok{(}\AttributeTok{data=}\NormalTok{ctr, ggplot2}\SpecialCharTok{::}\FunctionTok{aes}\NormalTok{(}\AttributeTok{x=}\NormalTok{Comp, }\AttributeTok{y=}\NormalTok{inertia)) }\SpecialCharTok{+}
\NormalTok{  ggplot2}\SpecialCharTok{::}\FunctionTok{geom\_bar}\NormalTok{(}\AttributeTok{stat =}\StringTok{"identity"}\NormalTok{, ggplot2}\SpecialCharTok{::}\FunctionTok{aes}\NormalTok{(}\AttributeTok{fill=}\FunctionTok{as.factor}\NormalTok{(inertia)), }\AttributeTok{show.legend =} \ConstantTok{FALSE}\NormalTok{) }\SpecialCharTok{+}
\NormalTok{  ggplot2}\SpecialCharTok{::}\FunctionTok{geom\_text}\NormalTok{(ggplot2}\SpecialCharTok{::}\FunctionTok{aes}\NormalTok{(}\AttributeTok{label=}\FunctionTok{paste0}\NormalTok{(}\FunctionTok{round}\NormalTok{(inertia,}\DecValTok{2}\NormalTok{),}\StringTok{"\%"}\NormalTok{),}
                                  \AttributeTok{y=}\NormalTok{inertia}\FloatTok{+0.1}\NormalTok{, }\AttributeTok{fontface=}\StringTok{\textquotesingle{}bold\textquotesingle{}}\NormalTok{),}
                                  \AttributeTok{vjust=}\DecValTok{0}\NormalTok{,}
                                  \AttributeTok{color=}\StringTok{"black"}\NormalTok{,}
                                  \AttributeTok{position =}\NormalTok{ ggplot2}\SpecialCharTok{::}\FunctionTok{position\_dodge}\NormalTok{(}\DecValTok{1}\NormalTok{), }\AttributeTok{size=}\FloatTok{4.5}\NormalTok{) }\SpecialCharTok{+}
\NormalTok{  ggplot2}\SpecialCharTok{::}\FunctionTok{theme\_classic}\NormalTok{() }\SpecialCharTok{+}
\NormalTok{  ggplot2}\SpecialCharTok{::}\FunctionTok{labs}\NormalTok{(}\AttributeTok{title=}\StringTok{"Barplot of variables\textquotesingle{} contribution"}\NormalTok{,}
                \AttributeTok{subtitle=}\FunctionTok{paste0}\NormalTok{(}\StringTok{"Using the first "}\NormalTok{,ndim,}\StringTok{" dimensions"}\NormalTok{),}
                \AttributeTok{caption=}\FunctionTok{paste0}\NormalTok{(}\StringTok{"The two first coordinates (dimensions) explain "}\NormalTok{,}
                               \FunctionTok{round}\NormalTok{(}\FunctionTok{sum}\NormalTok{(inertia[}\DecValTok{1}\SpecialCharTok{:}\DecValTok{2}\NormalTok{]),}\DecValTok{2}\NormalTok{),}\StringTok{"\% of the variability"}\NormalTok{)) }\SpecialCharTok{+}
\NormalTok{  ggplot2}\SpecialCharTok{::}\FunctionTok{theme}\NormalTok{(}
    \AttributeTok{text=}\NormalTok{ggplot2}\SpecialCharTok{::}\FunctionTok{element\_text}\NormalTok{(}\AttributeTok{family=}\StringTok{"serif"}\NormalTok{),}
    \AttributeTok{plot.title =}\NormalTok{ ggplot2}\SpecialCharTok{::}\FunctionTok{element\_text}\NormalTok{(}\AttributeTok{face=}\StringTok{"bold"}\NormalTok{, }\AttributeTok{size=}\DecValTok{14}\NormalTok{, }\AttributeTok{hjust =} \DecValTok{0}\NormalTok{),}
    \AttributeTok{axis.text=}\NormalTok{ggplot2}\SpecialCharTok{::}\FunctionTok{element\_text}\NormalTok{(}\AttributeTok{colour=}\StringTok{"black"}\NormalTok{, }\AttributeTok{size=}\DecValTok{12}\NormalTok{),}
    \AttributeTok{axis.title=}\NormalTok{ggplot2}\SpecialCharTok{::}\FunctionTok{element\_text}\NormalTok{(}\AttributeTok{colour=}\StringTok{"black"}\NormalTok{, }\AttributeTok{size=}\DecValTok{16}\NormalTok{,}\AttributeTok{face=}\StringTok{"bold"}\NormalTok{)}
\NormalTok{  ) }\SpecialCharTok{+}
\NormalTok{  ggplot2}\SpecialCharTok{::}\FunctionTok{xlab}\NormalTok{(}\StringTok{"Dimensions"}\NormalTok{) }\SpecialCharTok{+}\NormalTok{ ggplot2}\SpecialCharTok{::}\FunctionTok{ylab}\NormalTok{(}\StringTok{"\% of contribution"}\NormalTok{) }\SpecialCharTok{+}
\NormalTok{  ggplot2}\SpecialCharTok{::}\FunctionTok{scale\_fill\_brewer}\NormalTok{(}\AttributeTok{palette=}\StringTok{\textquotesingle{}Blues\textquotesingle{}}\NormalTok{) }\SpecialCharTok{+} 
\NormalTok{  ggplot2}\SpecialCharTok{::}\FunctionTok{scale\_x\_discrete}\NormalTok{(}\AttributeTok{limits=}\FunctionTok{paste0}\NormalTok{(}\StringTok{\textquotesingle{}Dim\textquotesingle{}}\NormalTok{,}\DecValTok{1}\SpecialCharTok{:}\NormalTok{ndim))}

\NormalTok{bplt}
\end{Highlighting}
\end{Shaded}

\hypertarget{collinearity}{%
\subsubsection{\texorpdfstring{\emph{Collinearity}}{Collinearity}}\label{collinearity}}

Then, we can decide to remove predictors highly correlated to others, by
looking at the Variance Inflation Factor (VIF) of each predictor. In the
literature, a VIF \textgreater{} 10 (or 5) usually indicate a
collinearity issue. We use the package \textbf{usdm} which has two
functions called \texttt{vifcor} and \texttt{vifstep} that implement
different strategies to deal with collinearity among variables.

\begin{Shaded}
\begin{Highlighting}[]

\CommentTok{\# detect collineary issues iteratively using a stepwise method}
\NormalTok{vif\_step\_method }\OtherTok{\textless{}{-}}\NormalTok{ usdm}\SpecialCharTok{::}\FunctionTok{vifstep}\NormalTok{(env\_data,}
             \AttributeTok{th =} \DecValTok{10}\NormalTok{, }\CommentTok{\# VIF threshold}
             \AttributeTok{maxobservations=}\DecValTok{6000}\NormalTok{) }\CommentTok{\# maximum number of grid cells to sample}

\CommentTok{\# detect collinearity issues using 2{-}by{-}2 correlations}
\NormalTok{vif\_cor\_method }\OtherTok{\textless{}{-}}\NormalTok{ usdm}\SpecialCharTok{::}\FunctionTok{vifcor}\NormalTok{(env\_data,}
             \AttributeTok{th =} \FloatTok{0.7}\NormalTok{, }\CommentTok{\# correlation threshold}
             \AttributeTok{maxobservations=}\DecValTok{6000}\NormalTok{) }

\CommentTok{\# select a method and exclude variables from the set of candidate predictors}
\NormalTok{env\_data\_selected }\OtherTok{\textless{}{-}}\NormalTok{ usdm}\SpecialCharTok{::}\FunctionTok{exclude}\NormalTok{(env\_data, }\AttributeTok{vif=}\NormalTok{vif\_cor\_method)}
\end{Highlighting}
\end{Shaded}

\begin{infobox}
We removed collinearity among our environmental predictors by using the
function \textbf{\texttt{vifcor}} which uses both VIF values and a
correlation threshold. The function detects pairs of highly correlated
variables (i.e.~with correlation \textgreater{} 0.7 for example), then
excludes the variable of the pair with the highest VIF value.

\end{infobox}

\hypertarget{ii.-model-training-evaluation}{%
\section{II. Model training \&
evaluation}\label{ii.-model-training-evaluation}}

We are going to train our model with the set of environmental predictors
previously selected and the maximum entropy algorithm MaxEnt (Phillips,
Anderson, and Schapire 2006).

\begin{Shaded}
\begin{Highlighting}[]
\CommentTok{\# if you do not have it yet, download the last version of Maxent (3.4.1)}
\CommentTok{\# It will be copied into the dismo/java folder}
\NormalTok{success }\OtherTok{\textless{}{-}}\NormalTok{ rmaxent}\SpecialCharTok{::}\FunctionTok{get\_maxent}\NormalTok{(}\AttributeTok{quiet=}\ConstantTok{TRUE}\NormalTok{)}
\NormalTok{success }\CommentTok{\# if \textless{} 0, the download/copy failed.}
\end{Highlighting}
\end{Shaded}

In \textbf{UsefulPlants}, MaxEnt is tuned and evaluated using the
\emph{masked geographically structured approach} (Radosavljevic and
Anderson 2014), which is a variant of the \emph{k-fold} cross-validation
that provides a better ability to detect over-fitting.

\hypertarget{data-partitioning}{%
\subsection{\texorpdfstring{\emph{Data
partitioning}}{Data partitioning}}\label{data-partitioning}}

We use the function \texttt{make\_geographic\_block} to spatially
segregate occurrence records into \emph{k=3} geographical bins with
approximately the same number of points.

\begin{Shaded}
\begin{Highlighting}[]

\CommentTok{\# (optional) rasterise the training area}
\NormalTok{raster\_template  }\OtherTok{\textless{}{-}} \FunctionTok{tryCatch}\NormalTok{(}
\NormalTok{  raster}\SpecialCharTok{::}\FunctionTok{raster}\NormalTok{(my\_training\_area, }\AttributeTok{res =}\NormalTok{ grid\_res),}
  \CommentTok{\# if any error occur, convert training area to \textquotesingle{}sp\textquotesingle{} object}
  \AttributeTok{error=}\ControlFlowTok{function}\NormalTok{(err)\{}
\NormalTok{    raster}\SpecialCharTok{::}\FunctionTok{raster}\NormalTok{(}\FunctionTok{as}\NormalTok{(my\_training\_area,}\StringTok{"Spatial"}\NormalTok{), }\AttributeTok{res =}\NormalTok{ grid\_res)}
\NormalTok{    \}}
\NormalTok{  )}
\NormalTok{bg }\OtherTok{\textless{}{-}}\NormalTok{ fasterize}\SpecialCharTok{::}\FunctionTok{fasterize}\NormalTok{(sf}\SpecialCharTok{::}\FunctionTok{st\_cast}\NormalTok{(domain,}\StringTok{"MULTIPOLYGON"}\NormalTok{),}
\NormalTok{                           raster\_template) }\SpecialCharTok{{-}} \DecValTok{1} \CommentTok{\# {-}1 to ensure background is a distinct block}

\CommentTok{\# read occurrence data}
\NormalTok{occ\_data }\OtherTok{=} \FunctionTok{read.csv}\NormalTok{(params}\SpecialCharTok{$}\NormalTok{occ\_file)}

\CommentTok{\# define the number of geographic blocks to create}
\NormalTok{K }\OtherTok{=} \DecValTok{3}

\CommentTok{\# Partition both occurrence records and training area into K geographic blocks}
\NormalTok{my\_geo\_blocks }\OtherTok{=} \FunctionTok{make\_geographic\_block}\NormalTok{(occ\_data, }\CommentTok{\# occurrence records data}
                      \AttributeTok{k =}\NormalTok{ K,                    }\CommentTok{\# number of geographic blocks}
                      \AttributeTok{bg =}\NormalTok{ bg,                  }\CommentTok{\# optional (i.e. can be left out)}
                      \AttributeTok{grid\_res =} \FloatTok{0.1666667}\NormalTok{,     }\CommentTok{\# resolution in decimal lat/lon}
                      \AttributeTok{sf=}\ConstantTok{TRUE}\NormalTok{,                  }\CommentTok{\# return a sf object}
                      \AttributeTok{verbose=}\ConstantTok{FALSE}\NormalTok{)            }\CommentTok{\# run quietly}
\end{Highlighting}
\end{Shaded}

\begin{infobox}

Internally, \texttt{make\_geographic\_block}:

\begin{itemize}
\item
  Used an unsupervised classification algorithm to cluster occurrence
  points into equal group size given their geographic
  coordinates/position
\item
  Built convex-hulls around each cluster
\item
  Made sure clusters do not overlap
\end{itemize}

\end{infobox}

\hypertarget{model-tuning}{%
\subsection{\texorpdfstring{\emph{Model
tuning}}{Model tuning}}\label{model-tuning}}

MaxEnt will be trained iteratively using \emph{k-1(=2)} bins and tested
in the last bin. At each iteration, we will compute a range of
evaluation metrics among: (i) the corrected Akaike Information criterion
(\(AIC_c\)), (ii) the tenth percentile of the training omission rate
(\(OR_{10}\)), the (iii) the Area Under the Curve of the receiver
operating characteristic (AUC), and (iv) the maximum of the True Skill
Statistics (TSS).

We are going to repeat this procedure for a range of \(\beta\)
regularization coefficients (hereafter called \(\beta\) multipliers) and
select the model with the \(\beta\) multiplier that obtain the best
overall performance given the evaluation metrics selected.

\begin{Shaded}
\begin{Highlighting}[]
\CommentTok{\# create a vector of beta multipliers to explore}
\NormalTok{beta\_mult }\OtherTok{=} \FunctionTok{c}\NormalTok{(}\DecValTok{1}\NormalTok{,}\DecValTok{6}\NormalTok{,}\DecValTok{10}\NormalTok{)}

\CommentTok{\# specify the evaluation metrics to compute}
\NormalTok{eval\_metrics}\OtherTok{=}\FunctionTok{c}\NormalTok{(}\StringTok{"auc"}\NormalTok{,           }\CommentTok{\# AUC}
               \StringTok{"omission\_rate"}\NormalTok{, }\CommentTok{\# OR10}
               \StringTok{"tss"}\NormalTok{,           }\CommentTok{\# TSS}
               \StringTok{"ic"}\NormalTok{)            }\CommentTok{\# AICc}

\CommentTok{\# specify a folder where the model outputs will be written}
\NormalTok{maxent\_output\_dir }\OtherTok{=} \FunctionTok{dirname}\NormalTok{(raster}\SpecialCharTok{::}\FunctionTok{rasterTmpFile}\NormalTok{())}
  
\CommentTok{\# create a list with maxent settings}
\NormalTok{maxent\_settings }\OtherTok{\textless{}{-}} \FunctionTok{list}\NormalTok{(}
  \AttributeTok{path\_to\_maxent            =} \FunctionTok{system.file}\NormalTok{(}\StringTok{\textquotesingle{}java/maxent.jar\textquotesingle{}}\NormalTok{, }\AttributeTok{package=}\StringTok{\textquotesingle{}dismo\textquotesingle{}}\NormalTok{),}
  \AttributeTok{visible                   =} \ConstantTok{FALSE}\NormalTok{,  }\CommentTok{\# hide MaxEnt GUI}
  \AttributeTok{writemess                 =} \ConstantTok{FALSE}\NormalTok{,  }\CommentTok{\# do not write MESS raster}
  \AttributeTok{writebackgroundpredictions=} \ConstantTok{TRUE}\NormalTok{,   }\CommentTok{\# write background predictions}
  \AttributeTok{maximumbackground         =} \DecValTok{50000}\NormalTok{,  }\CommentTok{\# sample up to 50,000 background points}
  \AttributeTok{betamultiplier            =}\NormalTok{ beta\_mult,}
  \AttributeTok{eval\_metrics              =}\NormalTok{ eval\_metrics,}
  \AttributeTok{prefixes                  =} \ConstantTok{FALSE}\NormalTok{,}
  \AttributeTok{threshold                 =} \ConstantTok{FALSE}\NormalTok{,  }\CommentTok{\# do not use threshold feature class}
  \AttributeTok{hinge                     =} \ConstantTok{TRUE}\NormalTok{,   }\CommentTok{\# use threshold feature class}
  \AttributeTok{outputformat              =} \StringTok{\textquotesingle{}raw\textquotesingle{}}\NormalTok{,  }\CommentTok{\# keep raw outputs (no transformation)}
  \AttributeTok{outputgrids               =} \ConstantTok{FALSE}   \CommentTok{\# do not write ascii grids}
\NormalTok{)}

\CommentTok{\# optionally compute a sampling bias prior layer to account for unevenly sampled}
\CommentTok{\# locations within the study area}

\CommentTok{\# get species coordinates (coordinates from higher taxonomic ranks or }
\CommentTok{\# other related species can also be included)}
\NormalTok{coords }\OtherTok{\textless{}{-}}\NormalTok{ occ\_data }\SpecialCharTok{\%\textgreater{}\%}
\NormalTok{  dplyr}\SpecialCharTok{::}\FunctionTok{select}\NormalTok{(}\FunctionTok{c}\NormalTok{(decimalLongitude, decimalLatitude))}

\CommentTok{\# count number of points per grid cell}
\NormalTok{rasterized\_occ }\OtherTok{\textless{}{-}}\NormalTok{ raster}\SpecialCharTok{::}\FunctionTok{rasterize}\NormalTok{(coords,}
\NormalTok{                                    bg,         }\CommentTok{\# rasterised training area}
                                    \AttributeTok{update=}\ConstantTok{TRUE}\NormalTok{,}
                                    \AttributeTok{fun=}\StringTok{\textquotesingle{}count\textquotesingle{}}\NormalTok{)}

\CommentTok{\# detect grid cells with presence locations}
\NormalTok{presences }\OtherTok{\textless{}{-}} \FunctionTok{which}\NormalTok{(}\SpecialCharTok{!}\FunctionTok{is.na}\NormalTok{(raster}\SpecialCharTok{::}\FunctionTok{values}\NormalTok{(rasterized\_occ)) }\SpecialCharTok{\&}
\NormalTok{                     raster}\SpecialCharTok{::}\FunctionTok{values}\NormalTok{(rasterized\_occ) }\SpecialCharTok{\textgreater{}}\NormalTok{ 0L)}
\NormalTok{pres\_locs }\OtherTok{\textless{}{-}}\NormalTok{ raster}\SpecialCharTok{::}\FunctionTok{coordinates}\NormalTok{(rasterized\_occ)[presences, ]}

\CommentTok{\# compute density}
\NormalTok{dens }\OtherTok{\textless{}{-}}\NormalTok{ MASS}\SpecialCharTok{::}\FunctionTok{kde2d}\NormalTok{(pres\_locs[,}\DecValTok{1}\NormalTok{], pres\_locs[,}\DecValTok{2}\NormalTok{],}
                    \AttributeTok{n =} \FunctionTok{c}\NormalTok{(}\FunctionTok{nrow}\NormalTok{(rasterized\_occ), }\FunctionTok{ncol}\NormalTok{(rasterized\_occ)),}
                    \AttributeTok{h=}\DecValTok{2}\NormalTok{) }\CommentTok{\# bandwidth}
\NormalTok{dens\_rast }\OtherTok{\textless{}{-}}\NormalTok{ raster}\SpecialCharTok{::}\FunctionTok{raster}\NormalTok{(dens)}

\CommentTok{\# make sure the raster matches the training area raster}
\NormalTok{biasrast }\OtherTok{\textless{}{-}}\NormalTok{ raster}\SpecialCharTok{::}\FunctionTok{resample}\NormalTok{(dens\_ras, bg, }\AttributeTok{method=}\StringTok{"ngb"}\NormalTok{)}

\CommentTok{\# save to temporary folder}
\NormalTok{sampbias\_fn }\OtherTok{\textless{}{-}} \FunctionTok{file.path}\NormalTok{(maxent\_output\_dir,}\StringTok{"sampbiasrast.tif"}\NormalTok{)}
\NormalTok{bias\_rast }\OtherTok{\textless{}{-}} \FunctionTok{maskCover}\NormalTok{(biasrast, my\_training\_area, }\AttributeTok{filename=}\NormalTok{sampbias\_fn)}
\CommentTok{\# rm(bias\_rast) if not needed anymore}

\CommentTok{\# add the path to the sampling bias file to the list of settings for maxent}
\NormalTok{maxent\_settings}\SpecialCharTok{$}\NormalTok{biasfile }\OtherTok{\textless{}{-}}\NormalTok{ sampbias\_fn}

\CommentTok{\# perform the block cross{-}validation}
\NormalTok{model\_output }\OtherTok{\textless{}{-}} \FunctionTok{block\_cv\_maxent}\NormalTok{(}
                            \CommentTok{\# our occurrence dataset}
                            \AttributeTok{loc\_dat =}\NormalTok{ occ\_data,}
                                
                            \CommentTok{\# raster stack of environmental predictors}
                            \AttributeTok{env\_dat =}\NormalTok{ env\_data\_selected,}
                            
                            \CommentTok{\# number of blocks}
                            \AttributeTok{k=}\NormalTok{K, }
                            
                            \CommentTok{\# optional coordinates headers}
                            \AttributeTok{coordHeaders=}\FunctionTok{c}\NormalTok{(}\StringTok{"decimalLongitude"}\NormalTok{,}
                                           \StringTok{"decimalLatitude"}\NormalTok{),}
                            
                            \CommentTok{\# our rasterised training area}
                            \AttributeTok{bg\_masks =}\NormalTok{ my\_geo\_blocks,}
                            
                            \CommentTok{\# output directory}
                            \AttributeTok{outputdir =}\NormalTok{ maxent\_output\_dir,}
                            
                            \CommentTok{\# (optional) species name}
                            \AttributeTok{species\_name =} \StringTok{"Abies\_spectabilis"}\NormalTok{,}
                            
                            \CommentTok{\# list of settings for Maxent}
                            \AttributeTok{maxent\_settings=}\NormalTok{maxent\_settings,}
                            
                            \CommentTok{\# name of the parameter that vary in maxent\_settings}
                            \CommentTok{\# list}
                            \AttributeTok{varying\_parameter\_name=}\StringTok{"betamultiplier"}\NormalTok{,}

                            \CommentTok{\# disable parallel computing}
                            \AttributeTok{do.parallel=}\ConstantTok{FALSE}\NormalTok{)}
\end{Highlighting}
\end{Shaded}

The function \texttt{block\_cv\_maxent} returns a data.frame
(\texttt{model\_output}) that reports the score of each evaluation
metric within each bin tested. Let's have a look at this:

Now, we can use the function \texttt{get\_best\_maxent\_model} to select
the best overall model.

\begin{Shaded}
\begin{Highlighting}[]
\NormalTok{best\_model }\OtherTok{\textless{}{-}} \FunctionTok{get\_best\_maxent}\NormalTok{(model\_output, }\AttributeTok{eval\_metrics=}\NormalTok{eval\_metrics)}
\end{Highlighting}
\end{Shaded}

\begin{infobox}

Internally, \textbf{\texttt{get\_best\_maxent\_model}}:

\begin{itemize}
\item
  Average evaluation metrics of each model across geographic bins
\item
  Sort the models from the lowest information criteria
  \rightarrow lowest \(OR_{10}\) \rightarrow highest AUC
  \rightarrow highest TSS.
\end{itemize}

\end{infobox}

\hypertarget{iii.-model-projection}{%
\section{III. Model projection}\label{iii.-model-projection}}

Now that we determined the best \(\beta\) multiplier for our model, we
can train it using the full dataset (i.e.~without withholding occurrence
records) and project it across space.

\hypertarget{references}{%
\section*{References}\label{references}}
\addcontentsline{toc}{section}{References}

\hypertarget{refs}{}
\begin{CSLReferences}{1}{0}
\leavevmode\vadjust pre{\hypertarget{ref-ade4}{}}%
Dray, Stéphane, Anne-Béatrice Dufour, and Daniel Chessel. 2007. {``The
{ade4} Package -- {II}: Two-Table and {K}-Table Methods.''} \emph{R
News} 7 (2): 47--52. \url{https://cran.r-project.org/doc/Rnews/}.

\leavevmode\vadjust pre{\hypertarget{ref-geodata}{}}%
Hijmans, Robert J., Márcia Barbosa, Aniruddha Ghosh, and Alex Mandel.
2023. \emph{Geodata: Download Geographic Data}.
\url{https://CRAN.R-project.org/package=geodata}.

\leavevmode\vadjust pre{\hypertarget{ref-dismo}{}}%
Hijmans, Robert J., Steven Phillips, John Leathwick, and Jane Elith.
2021. \emph{Dismo: Species Distribution Modeling}.
\url{https://CRAN.R-project.org/package=dismo}.

\leavevmode\vadjust pre{\hypertarget{ref-usdm}{}}%
Naimi, Babak, Nicholas a.s. Hamm, Thomas A. Groen, Andrew K. Skidmore,
and Albertus G. Toxopeus. 2014. {``Where Is Positional Uncertainty a
Problem for Species Distribution Modelling.''} \emph{Ecography} 37:
191--203. \url{https://doi.org/10.1111/j.1600-0587.2013.00205.x}.

\leavevmode\vadjust pre{\hypertarget{ref-MaxEntPhillips2006}{}}%
Phillips, Steven J, Robert P Anderson, and Robert E Schapire. 2006.
{``Maximum Entropy Modeling of Species Geographic Distributions.''}
\emph{Ecological Modelling} 190 (3-4): 231--59.

\leavevmode\vadjust pre{\hypertarget{ref-radosavljevic2014}{}}%
Radosavljevic, Aleksandar, and Robert P Anderson. 2014. {``Making Better
Maxent Models of Species Distributions: Complexity, Overfitting and
Evaluation.''} \emph{Journal of Biogeography} 41 (4): 629--43.

\leavevmode\vadjust pre{\hypertarget{ref-ggplot2}{}}%
Wickham, Hadley. 2016. \emph{Ggplot2: Elegant Graphics for Data
Analysis}. Springer-Verlag New York.
\url{https://ggplot2.tidyverse.org}.

\leavevmode\vadjust pre{\hypertarget{ref-dplyr}{}}%
Wickham, Hadley, Romain François, Lionel Henry, Kirill Müller, and Davis
Vaughan. 2023. \emph{Dplyr: A Grammar of Data Manipulation}.
\url{https://CRAN.R-project.org/package=dplyr}.

\leavevmode\vadjust pre{\hypertarget{ref-kableExtra}{}}%
Zhu, Hao. 2023. \emph{kableExtra: Construct Complex Table with 'Kable'
and Pipe Syntax}.

\end{CSLReferences}

\end{document}
